%% Template for a CV
%% Author: Rob J Hyndman

\documentclass[10pt,a4paper,]{article}
\usepackage[scaled=0.86]{DejaVuSansMono}
\usepackage[sfdefault,lf,t]{carlito}

% Change color to blue
\usepackage{color,xcolor}
\definecolor{headcolor}{HTML}{990000}

\usepackage{ifxetex,ifluatex}
\usepackage{fixltx2e} % provides \textsubscript
\ifnum 0\ifxetex 1\fi\ifluatex 1\fi=0 % if pdftex
  \usepackage[T1]{fontenc}
  \usepackage[utf8]{inputenc}
\else % if luatex or xelatex
  \ifxetex
    \usepackage{mathspec}
  \else
    \usepackage{fontspec}
  \fi
  \defaultfontfeatures{Ligatures=TeX,Scale=MatchLowercase}
\fi

\usepackage[utf8]{inputenc}
\usepackage[T1]{fontenc}

% use upquote if available, for straight quotes in verbatim environments
\IfFileExists{upquote.sty}{\usepackage{upquote}}{}
% use microtype if available
\IfFileExists{microtype.sty}{%
\usepackage[]{microtype}
\UseMicrotypeSet[protrusion]{basicmath} % disable protrusion for tt fonts
}{}
\PassOptionsToPackage{hyphens}{url} % url is loaded by hyperref
\usepackage[unicode=true,hidelinks]{hyperref}
\urlstyle{same}  % don't use monospace font for urls
\usepackage{geometry}
\geometry{left=1.75cm,right=1.75cm,top=2.2cm,bottom=2cm}

\usepackage{longtable,booktabs}
% Fix footnotes in tables (requires footnote package)
\IfFileExists{footnote.sty}{\usepackage{footnote}\makesavenoteenv{long table}}{}
\IfFileExists{parskip.sty}{%
\usepackage{parskip}
}{% else
\setlength{\parindent}{0pt}
\setlength{\parskip}{6pt plus 2pt minus 1pt}
}
\setlength{\emergencystretch}{3em}  % prevent overfull lines
\providecommand{\tightlist}{%
  \setlength{\itemsep}{0pt}\setlength{\parskip}{0pt}}
\setcounter{secnumdepth}{0}

% set default figure placement to htbp
\makeatletter
\def\fps@figure{htbp}
\makeatother



\date{April 2023}

\definecolor{headcolor}{HTML}{1E49BF}
\definecolor{linkscolor}{HTML}{026dbb}
\hypersetup{colorlinks=true, linkcolor=linkscolor, filecolor=linkscolor, urlcolor=linkscolor, urlbordercolor=linkscolor}

\usepackage{paralist,ragged2e,datetime}
\usepackage[hyphens]{url}
\usepackage{fancyhdr,enumitem,pifont}
\usepackage[compact,small,sf,bf]{titlesec}

\RaggedRight
\sloppy

% Header and footer
\pagestyle{fancy}
\makeatletter
\lhead{\sf\textcolor[gray]{0.4}{Curriculum Vitae: \@name}}
\rhead{\sf\textcolor[gray]{0.4}{\thepage}}
\cfoot{}
\def\headrule{{\color[gray]{0.4}\hrule\@height\headrulewidth\@width\headwidth \vskip-\headrulewidth}}
\makeatother

% Header box
\usepackage{tabularx}

\makeatletter
\def\name#1{\def\@name{#1}}
\def\info#1{\def\@info{#1}}
\makeatother
\newcommand{\shadebox}[3][.9]{\fcolorbox[gray]{0}{#1}{\parbox{#2}{#3}}}

\usepackage{calc}
\newlength{\headerboxwidth}
\setlength{\headerboxwidth}{\textwidth}
%\addtolength{\headerboxwidth}{0.2cm}
\makeatletter
\def\maketitle{
\thispagestyle{plain}
\vspace*{-1.4cm}
\shadebox[0.9]{\headerboxwidth}{\sf\color{headcolor}\hfil
\hbox to 0.98\textwidth{\begin{tabular}{l}
\\[-0.3cm]
\LARGE\textbf{\@name}
\\[0.1cm]\large Postdoctoral Researcher\\[0.6cm]
\normalsize\textbf{Curriculum Vitae}\\
\normalsize April 2023
\end{tabular}
\hfill\hbox{\fontsize{9}{12}\sf
\begin{tabular}{@{}rl@{}}
\@info
\end{tabular}}}\hfil
}
\vspace*{0.2cm}}
\makeatother

% Section headings
\titlelabel{}
\titlespacing{\section}{0pt}{1.5ex}{0.5ex}
\titleformat*{\section}{\color{headcolor}\large\sf\bfseries}
\titleformat*{\subsection}{\color{headcolor}\sf\bfseries}
\titlespacing{\subsection}{0pt}{1ex}{0.5ex}

% Miscellaneous dimensions
\setlength{\parskip}{0ex}
\setlength{\parindent}{0em}
\setlength{\headheight}{15pt}
\setlength{\tabcolsep}{0.15cm}
\clubpenalty = 10000
\widowpenalty = 10000
\setlist{itemsep=1pt}
\setdescription{labelwidth=1.2cm,leftmargin=1.5cm,labelindent=1.5cm,font=\rm}

% Make nicer bullets
\renewcommand{\labelitemi}{\ding{228}}

\usepackage{booktabs,fontawesome5}
%\usepackage[t1,scale=0.86]{sourcecodepro}

\name{Matthias Grenié}
\def\imagetop#1{\vtop{\null\hbox{#1}}}
\info{%
\raisebox{-0.05cm}{\imagetop{\faIcon{map-marker-alt}}} &  \imagetop{\begin{tabular}{@{}l@{}}German
Centre for Integrative Biodversity Research (iDiv),
Halle-Jena-Leipzig\end{tabular}}\\ %
\faIcon{home} & \href{http://rekyt.github.io}{rekyt.github.io}\\% %
%
\faIcon{phone-alt} & +49 341 9733176\\%
\faIcon{envelope} & \href{mailto:matthias.grenie@idiv.de}{\nolinkurl{matthias.grenie@idiv.de}}\\%
\faIcon{twitter} & \href{https://twitter.com/LeNematode}{@LeNematode}\\%
\faIcon{github} & \href{https://github.com/Rekyt}{Rekyt}\\%
%
\faIcon{google} & \href{https://scholar.google.com/citations?user=fZ1\_d7QAAAAJ}{fZ1\_d7QAAAAJ}\\%
\faIcon{researchgate} & \href{https://www.researchgate.net/profile/Matthias-Grenie}{Matthias-Grenie}\\%
\faIcon{orcid} & \href{https://orcid.org/0000-0002-4659-7522}{0000-0002-4659-7522}\\%
}


%\usepackage{inconsolata}


\setlength\LTleft{0pt}
\setlength\LTright{0pt}

% Pandoc CSL macros
\newlength{\cslhangindent}
\setlength{\cslhangindent}{1.5em}
\newlength{\csllabelwidth}
\setlength{\csllabelwidth}{3em}
\newenvironment{CSLReferences}[3] % #1 hanging-ident, #2 entry spacing
 {% don't indent paragraphs
  \setlength{\parindent}{0pt}
  % turn on hanging indent if param 1 is 1
  \ifodd #1 \everypar{\setlength{\hangindent}{\cslhangindent}}\ignorespaces\fi
  % set entry spacing
  \ifnum #2 > 0
  \setlength{\parskip}{#2\baselineskip}
  \fi
 }%
 {}
\usepackage{calc}
\newcommand{\CSLBlock}[1]{#1\hfill\break}
\newcommand{\CSLLeftMargin}[1]{\parbox[t]{\csllabelwidth}{\hfill #1~}}
\newcommand{\CSLRightInline}[1]{\parbox[t]{\linewidth - \cslhangindent - \csllabelwidth}{#1}\vspace{0.8ex}}
\newcommand{\CSLIndent}[1]{\hspace{\cslhangindent}#1}


\def\endfirstpage{\newpage}

\begin{document}
\maketitle


\hypertarget{academic-education}{%
\section{Academic Education}\label{academic-education}}

\begin{longtable}{@{\extracolsep{\fill}}ll}
2016-2020 & \parbox[t]{0.85\textwidth}{%
\textbf{PhD in Ecology}\hfill{\footnotesize École Doctorale 584 GAIA, Univ. Montpellier}\newline
  Montpellier, France\par%
  \vspace{0.1cm}\begin{minipage}{0.7\textwidth}%
\begin{itemize}%
\item Supervised by \textbf{Pr. Françoiz Munoz} and \textbf{Dr. Cyrille Violle}\\~Title: 'Outside the norm: deviation from ecological optimality and functional originality'%
\end{itemize}%
\end{minipage}%
\vspace{\parsep}}\\
2013-2015 & \parbox[t]{0.85\textwidth}{%
\textbf{MSc. in General Biology}\hfill{\footnotesize École Normale Supérieure de Lyon}\newline
  Lyon, France\par%
  \empty%
\vspace{\parsep}}\\
2012-2013 & \parbox[t]{0.85\textwidth}{%
\textbf{Bsc. in General Biology}\hfill{\footnotesize École Normale Supérieure de Lyon}\newline
  Lyon, France\par%
  \empty%
\vspace{\parsep}}\\
2010-2012 & \parbox[t]{0.85\textwidth}{%
\textbf{Prep. Classes for 'Grandes Écoles'}\hfill{\footnotesize Lycée Saint-Louis}\newline
  Paris, France\par%
  \vspace{0.1cm}\begin{minipage}{0.7\textwidth}%
\begin{itemize}%
\item Biology, Geology, Mathematics, Physics, and Chemistry%
\end{itemize}%
\end{minipage}%
\vspace{\parsep}}\\
\end{longtable}

\hypertarget{academic-experience}{%
\section{Academic Experience}\label{academic-experience}}

\begin{longtable}{@{\extracolsep{\fill}}ll}
Oct. 2020-\textbf{present} & \parbox[t]{0.85\textwidth}{%
\textbf{Postdoctoral Researcher}\hfill{\footnotesize German Center for Integrative Biodiversity Research (iDiv) / Leipzig University}\newline
  Leipzig, Germany\par%
  \vspace{0.1cm}\begin{minipage}{0.7\textwidth}%
\begin{itemize}%
\item Supervised by \textbf{Dr. Marten Winter} on the functional biogegraphy of alien plants%
\end{itemize}%
\end{minipage}%
\vspace{\parsep}}\\
2016-2020 & \parbox[t]{0.85\textwidth}{%
\textbf{PhD Student}\hfill{\footnotesize Univ. Montpellier / CEFE}\newline
  Montpellier, France\par%
  \vspace{0.1cm}\begin{minipage}{0.7\textwidth}%
\begin{itemize}%
\item Supervisors:  \textbf{Pr. François Munoz} and\textbf{Dr. Cyrille Violle}. Independent PhD funding%
\end{itemize}%
\end{minipage}%
\vspace{\parsep}}\\
Feb. 2015-June 2015 & \parbox[t]{0.85\textwidth}{%
\textbf{Master Intern}\hfill{\footnotesize EcoFOG}\newline
  Kourou, French Guiana, France\par%
  \vspace{0.1cm}\begin{minipage}{0.7\textwidth}%
\begin{itemize}%
\item Supervisor: \textbf{Dr. Bruno Hérault} on intraspecific variability in tropical trees growth in function of functional traits%
\end{itemize}%
\end{minipage}%
\vspace{\parsep}}\\
Sept. 2014-Dec. 2014 & \parbox[t]{0.85\textwidth}{%
\textbf{Master Intern}\hfill{\footnotesize ISEM, University of Montpellier}\newline
  Montpellier, France\par%
  \vspace{0.1cm}\begin{minipage}{0.7\textwidth}%
\begin{itemize}%
\item Supervisor: \textbf{Dr. Ophélie Ronce} on matrix population models%
\end{itemize}%
\end{minipage}%
\vspace{\parsep}}\\
Feb. 2014-July 2014 & \parbox[t]{0.85\textwidth}{%
\textbf{Master Intern}\hfill{\footnotesize Indiana University}\newline
  Bloomington, IN, USA\par%
  \vspace{0.1cm}\begin{minipage}{0.7\textwidth}%
\begin{itemize}%
\item Supervised by \textbf{Dr. Jean-François Goût} in Michael Lynch's lab on trait motifs in the \textit{Paramecium} complex%
\end{itemize}%
\end{minipage}%
\vspace{\parsep}}\\
June 2013-Aug. 2013 & \parbox[t]{0.85\textwidth}{%
\textbf{Bachelor Intern}\hfill{\footnotesize Université Pierre et Marie Curie Paris 6}\newline
  Paris, France\par%
  \vspace{0.1cm}\begin{minipage}{0.7\textwidth}%
\begin{itemize}%
\item Supervisors: \textbf{Dr. Éric Bapteste} and \textbf{Prof. Philippe Lopez} on sequence similarity networks of the metabolic pathways of \textit{Chlamydomonas reinhardtii} in function of their origin%
\end{itemize}%
\end{minipage}%
\vspace{\parsep}}\\
\end{longtable}

\hypertarget{scientific-supervision}{%
\section{Scientific Supervision}\label{scientific-supervision}}

\begin{longtable}{@{\extracolsep{\fill}}ll}
Apr. 2018 - May 2018 & \parbox[t]{0.85\textwidth}{%
\textbf{Supervised First year Master's Student Charlotte Guérineau}\\[-0.1cm]{\footnotesize Internship on functional rarity of diverse taxonomic groups (birds, plants, fishes, and soil bacteria) over France}}\\[0.4cm]
Apr. 2019 - June 2019 & \parbox[t]{0.85\textwidth}{%
\textbf{Co-supervised First year Master's Student Nathan Mazet}\\[-0.1cm]{\footnotesize Internship on global dietary strategies for birds, main supervisor: Jean-Yves Barnagaud}}\\[0.4cm]
\end{longtable}

\hypertarget{scientific-service}{%
\section{Scientific Service}\label{scientific-service}}

\begin{longtable}{@{\extracolsep{\fill}}ll}
2017-\textbf{now} & \parbox[t]{0.85\textwidth}{%
\textbf{Reviewer for Scientific Journals (\href{https://publons.com/wos-op/researcher/1466931/matthias-grenie}{see Publons profile})}\\[-0.1cm]{\footnotesize For \textit{Ecology Letters}, \textit{Functional Ecology}, \textit{Biological Reviews}, \textit{PeerJ}, \textit{Biological Conservation}, and \textit{Journal of Biogeography}}}\\[0.4cm]
2022-\textbf{now} & \parbox[t]{0.85\textwidth}{%
\textbf{Software reviewer for rOpenSci}\\[-0.1cm]{\footnotesize For the \href{https://github.com/ropensci/software-review/issues/505}{\texttt{npi} package}}}\\[0.4cm]
2022-\textbf{now} & \parbox[t]{0.85\textwidth}{%
\textbf{Post-doctoral researchers representative}\\[-0.1cm]{\footnotesize \href{https://www.idiv.de/en/council.html}{\textbf{iDiv Council}}, an advisory council of iDiv}}\\[0.4cm]
2021-\textbf{now} & \parbox[t]{0.85\textwidth}{%
\textbf{Organizer of the biweekly sDiv lab meetings}\\[-0.1cm]{\footnotesize sDiv team}}\\[0.4cm]
2022-\textbf{now} & \parbox[t]{0.85\textwidth}{%
\textbf{Member of PhD Advisory Committee of \href{https://www.idiv.de/en/profile/1248.html}{\textit{Rachel Souza Fereira}}}\\[-0.1cm]{\footnotesize }}\\[0.4cm]
2018-2020 & \parbox[t]{0.85\textwidth}{%
\textbf{Representative of PhD students in the CEFE lab council}\\[-0.1cm]{\footnotesize CEFE lab council, the main governing body of the lab}}\\[0.4cm]
2018 & \parbox[t]{0.85\textwidth}{%
\textbf{Co-organizer of half-day workshop on Open Access Practices}\\[-0.1cm]{\footnotesize An event on how to make your articles accessible, organized for \textasciitilde{}50 people of the CEFE in Montpellier}}\\[0.4cm]
2017 & \parbox[t]{0.85\textwidth}{%
\textbf{Co-organizer of Junior Conference \href{https://web.archive.org/web/20180823012054/http://www.mee.univ-montp2.fr/editions-precedentes/edition-2017/organisateurs-2017/}{'Models in Ecology and Evolution'}}\\[-0.1cm]{\footnotesize A junior conference organized yearly in Montpellier with \textasciitilde{}100 people attending from Masters to Post-docs}}\\[0.4cm]
\end{longtable}

\hypertarget{scientific-projects}{%
\section{Scientific Projects}\label{scientific-projects}}

\begin{longtable}{@{\extracolsep{\fill}}ll}
2017-\textbf{now} & \parbox[t]{0.85\textwidth}{%
\textbf{\href{https://www.fondationbiodiversite.fr/la-frb-en-action/programmes-et-projets/le-cesab/free/}{\textit{F}unctional \textit{R}arity in \textit{E}cology and \textit{E}volution group} (FREE)}\hfill{\footnotesize CESAB (French Biodiversity Synthesis Centre)}\newline
  \empty%
  \vspace{0.1cm}\begin{minipage}{0.7\textwidth}%
\begin{itemize}%
\item Core member of the group as the developer of the \texttt{funrar} package that computes functional rarity indices. Author of several core papers for the group. Led 3 papers attached to the work of the group.%
\end{itemize}%
\end{minipage}%
\vspace{\parsep}}\\
2021-\textbf{now} & \parbox[t]{0.85\textwidth}{%
\textbf{\href{https://glonaf.org/}{Global Naturalized Alien Flora} (GloNAF)}\hfill{\footnotesize GloNAF}\newline
  \empty%
  \vspace{0.1cm}\begin{minipage}{0.7\textwidth}%
\begin{itemize}%
\item Contributor to the regular meetings. Currently leading one article for the group.%
\end{itemize}%
\end{minipage}%
\vspace{\parsep}}\\
2022-\textbf{now} & \parbox[t]{0.85\textwidth}{%
\textbf{\href{https://www.fondationbiodiversite.fr/la-frb-en-action/programmes-et-projets/le-cesab/impacts/}{IMPACTS group}}\hfill{\footnotesize CESAB}\newline
  \empty%
  \vspace{0.1cm}\begin{minipage}{0.7\textwidth}%
\begin{itemize}%
\item Scientific Contributor to the group. Co-head of data management for the group. First meeting November 2022.%
\end{itemize}%
\end{minipage}%
\vspace{\parsep}}\\
\end{longtable}

\hypertarget{scientific-contributions}{%
\section{Scientific Contributions}\label{scientific-contributions}}

\faFile*~14 publications (5 as first author). \faQuoteLeft~457 citations
(Google Scholar). \faHSquare~h-index 10 (Google Scholar). \faRProject~6
R packages

\hypertarget{publications}{%
\subsection{Publications}\label{publications}}

\hypertarget{accepted}{%
\subsubsection{Accepted}\label{accepted}}

\hypertarget{bibliography}{}
\leavevmode\vadjust pre{\hypertarget{ref-Munoz_ecological_2023}{}}%
1. Munoz, F., Klausmeier, C., \ldots, \textbf{Grenié, M.}, Loiseau, N.,
Mahaut, L., Maire, A., Mouillot, D., Violle, C., \& Kraft, N. (2023).
The ecological causes of functional distinctiveness in communities.
\emph{Accepted in Ecology Letters}.
\url{https://doi.org/10.22541/au.166488862.28762630/v1}

\hypertarget{published}{%
\subsubsection{Published}\label{published}}

\hypertarget{bibliography}{}
\leavevmode\vadjust pre{\hypertarget{ref-Cutts_Links_2023}{}}%
1. Cutts, V., Hanz, D. M., Barajas-Barbosa, M. P., Schrodt, F.,
Steinbauer, M. J., Beierkuhnlein, C., Denelle, P., Fernández-Palacios,
J. M., Gaüzère, P., \textbf{Grenié, M.}, Irl, S. D. H., \ldots{} Algar,
A. C. (2023). Links to rare climates do not translate into distinct
traits for island endemics. \emph{Ecology Letters}, \emph{n/a}(n/a).
\url{https://doi.org/10.1111/ele.14169}

\leavevmode\vadjust pre{\hypertarget{ref-Grenie_fundiversity_2023}{}}%
2. \textbf{Grenié, M.}, \& Gruson, H. (2023). Fundiversity: A modular R
package to compute functional diversity indices. \emph{Ecography},
\emph{n/a}(n/a), e06585. \url{https://doi.org/10.1111/ecog.06585}

\leavevmode\vadjust pre{\hypertarget{ref-Gauzere_functional_2022}{}}%
3. Gaüzère, P., Denelle, P., Fournier, B., \textbf{Grenié, M.},
Delalandre, L., Münkemüller, T., Munoz, F., Violle, C., \& Thuiller, W.
(2022). The functional trait distinctiveness of plant species is scale
dependent. \emph{Ecography}, \emph{in press}.

\leavevmode\vadjust pre{\hypertarget{ref-Grenie_Harmonizing_2022}{}}%
4. \textbf{Grenié, M.}, Berti, E., Carvajal-Quintero, J., Dädlow, G. M.
L., Sagouis, A., \& Winter, M. (2022). Harmonizing taxon names in
biodiversity data: A review of tools, databases and best practices.
\emph{Methods in Ecology and Evolution}, 2041--210X.13802.
\url{https://doi.org/10.1111/2041-210X.13802}

\leavevmode\vadjust pre{\hypertarget{ref-Jurburg_community_2022}{}}%
5. Jurburg, S. D., Buscot, F., Chatzinotas, A., Chaudhari, N. M., Clark,
A. T., Garbowski, M., \textbf{Grenié, M.}, Hom, E. F. Y., Karakoç, C.,
Marr, S., Neumann, S., \ldots{} Heintz-Buschart, A. (2022). The
community ecology perspective of omics data. \emph{Microbiome},
\emph{10}(1), 225. \url{https://doi.org/10.1186/s40168-022-01423-8}

\leavevmode\vadjust pre{\hypertarget{ref-Mouillot_dimensionality_2021}{}}%
6. Mouillot, D., Loiseau, N., \textbf{Grenié, M.}, Algar, A. C.,
Allegra, M., Cadotte, M. W., Casajus, N., Denelle, P., Guéguen, M.,
Maire, A., Maitner, B., \ldots{} Auber, A. (2021). The dimensionality
and structure of species trait spaces. \emph{Ecology Letters},
\emph{24}(9), 1988--2009. \url{https://doi.org/10.1111/ele.13778}

\leavevmode\vadjust pre{\hypertarget{ref-Grenie_prediction_2020}{}}%
7. \textbf{Grenié, M.}, Violle, C., \& Munoz, F. (2020). Is prediction
of species richness from stacked species distribution models biased by
habitat saturation? \emph{Ecological Indicators}, \emph{111}, 105970.
\url{https://doi.org/10.1016/j.ecolind.2019.105970}

\leavevmode\vadjust pre{\hypertarget{ref-Loiseau_Global_2020}{}}%
8. Loiseau, N., Mouquet, N., Casajus, N., \textbf{Grenié, M.}, Guéguen,
M., Maitner, B., Mouillot, D., Ostling, A., Renaud, J., Tucker, C.,
Velez, L., \ldots{} Violle, C. (2020). Global distribution and
conservation status of ecologically rare mammal and bird species.
\emph{Nature Communications}, \emph{11}(1, 1), 5071.
\url{https://doi.org/10.1038/s41467-020-18779-w}

\leavevmode\vadjust pre{\hypertarget{ref-Barnagaud_Functional_2019}{}}%
9. Barnagaud, J.-Y., Mazet, N., Munoz, F., \textbf{Grenié, M.}, Denelle,
P., Sobral, M., Kissling, W. D., Sekercioglu, Ç. H., \& Violle, C.
(2019). Functional biogeography of dietary strategies in birds.
\emph{Global Ecology and Biogeography}, \emph{28}(7), 1004--1017.
\url{https://doi.org/10.1111/geb.12910}

\leavevmode\vadjust pre{\hypertarget{ref-Joffard_Effect_2019}{}}%
10. Joffard, N., Massol, F., \textbf{Grenié, M.}, Montgelard, C., \&
Schatz, B. (2019). Effect of pollination strategy, phylogeny and
distribution on pollination niches of Euro-Mediterranean orchids.
\emph{Journal of Ecology}, \emph{107}(1), 478--490.
\url{https://doi.org/10.1111/1365-2745.13013}

\leavevmode\vadjust pre{\hypertarget{ref-Grenie_Functional_2018}{}}%
11. \textbf{Grenié, M.}, Mouillot, D., Villéger, S., Denelle, P.,
Tucker, C. M., Munoz, F., \& Violle, C. (2018). Functional rarity of
coral reef fishes at the global scale: Hotspots and challenges for
conservation. \emph{Biological Conservation}, \emph{226}, 288--299.
\url{https://doi.org/10.1016/j.biocon.2018.08.011}

\leavevmode\vadjust pre{\hypertarget{ref-Munoz_ecolottery_2018}{}}%
12. Munoz, F., \textbf{Grenié, M.}, Denelle, P., Taudière, A., Laroche,
F., Tucker, C., \& Violle, C. (2018). Ecolottery: Simulating and
assessing community assembly with environmental filtering and neutral
dynamics in R. \emph{Methods in Ecology and Evolution}, \emph{9}(3),
693--703. \url{https://doi.org/10.1111/2041-210X.12918}

\leavevmode\vadjust pre{\hypertarget{ref-Grenie_funrar_2017}{}}%
13. \textbf{Grenié, M.}, Denelle, P., Tucker, C. M., Munoz, F., \&
Violle, C. (2017). Funrar: An R package to characterize functional
rarity. \emph{Diversity and Distributions}, \emph{23}(12), 1365--1371.
\url{https://doi.org/10.1111/ddi.12629}

\leavevmode\vadjust pre{\hypertarget{ref-Violle_Common_2017}{}}%
14. Violle, C., Thuiller, W., Mouquet, N., Munoz, F., Kraft, N. J. B.,
Cadotte, M. W., Livingstone, S. W., \textbf{Grenié, M.}, \& Mouillot, D.
(2017). A Common Toolbox to Understand, Monitor or Manage Rarity? A
Response to Carmona et al. \emph{Trends in Ecology \& Evolution},
\emph{32}(12), 891--893.
\url{https://doi.org/10.1016/j.tree.2017.10.001}

\hypertarget{r-packages}{%
\subsection{R packages}\label{r-packages}}

\begin{itemize}
\item
  \textbf{\texttt{funrar}} (main creator and maintainer)
  \hfill\break computes functional originality indices central for CESAB
  FREE group analyses
  \hfill\break ~\href{https://doi.org/10.1111/ddi.12629}{\faFile*~published
  paper} --
  \href{https://cran.r-project.org/package=funrar}{\faRProject~CRAN} --
  \href{https://github.com/Rekyt/funrar}{\faGithub~GitHub}
\item
  \textbf{\texttt{ecolottery}} (co-creator) \hfill\break simulates
  community assembly processes cleverly with both niche and neutral
  processes
  \hfill\break ~\href{https://doi.org/10.1111/2041-210X.12918}{\faFile*~published
  paper} --
  \href{https://cran.r-project.org/package=ecolottery}{\faRProject~CRAN}
  -- \href{https://github.com/frmunoz/ecolottery}{\faGithub~GitHub}
\item
  \textbf{\texttt{fundiversity}} (co-creator and maintainer)
  \hfill\break a fast and modular package to compute functional
  diversity indices
  \hfill\break ~\href{https:////doi.org/10.1111/ecog.06585}{\faFile*~published
  paper} --
  \href{https://cran.r-project.org/package=fundiversity}{\faRProject~CRAN}
  -- \href{https://github.com/bisaloo/fundiversity}{\faGithub~GitHub}
\item
  \textbf{\texttt{rromeo}} (co-creator and maintainer)
  \hfill\break accesses the
  \href{https://v2.sherpa.ac.uk/romeo/}{Sherpa/RoMEO API} that indexes
  the open access policies of journals
  \hfill\break~\href{https://docs.ropensci.org/rromeo}{published by
  rOpenSci} --
  \href{https://cran.r-project.org/package=rromeo}{\faRProject~CRAN} --
  \href{https://github.com/ropensci/rromeo}{\faGithub~GitHub}
\item
  \textbf{\texttt{funbiogeo}} (co-creator and maintainer)
  \hfill\break helps create functional biogeography workflows, developed
  in the context of the FREE CESAB group
  \hfill\break ~\href{https://github.com/FRBCesab/funbiogeo}{\faGithub~GitHub}
\item
  \textbf{\texttt{rtaxref}} (creator and maintainer)
  \hfill\break accesses the French Taxonomic Referential (TAXREF) API
  \hfill\break ~\href{https://github.com/Rekyt/rtaxref}{\faGithub~GitHub}
\item
  Significant contributions to other packages:
  \href{https://cran.r-project.org/package=taxize}{\textbf{\texttt{taxize}}},
  \href{https://cran.r-project.org/package=ggfortify}{\textbf{\texttt{ggfortify}}},
  and
  \href{https://cran.r-project.org/package=traitdataform}{\textbf{\texttt{traitdataform}}}
\end{itemize}

\hypertarget{conference-proceedings}{%
\subsection{Conference Proceedings}\label{conference-proceedings}}

\hypertarget{bibliography}{}
\leavevmode\vadjust pre{\hypertarget{ref-Grenie_Matching_2021}{}}%
1. \textbf{Grenié, M.}, Berti, E., Carvajal-Quintero, J., Winter, M., \&
Sagouis, A. (2021). Matching Species Names Across Biodiversity
Databases: Sources, tools, pitfalls and best practices for taxonomic
harmonization. \emph{Biodiversity Information Science and Standards},
\emph{5}, e75359. \url{https://doi.org/10.3897/biss.5.75359}

\hypertarget{invited-talks}{%
\subsection{Invited Talks}\label{invited-talks}}

\begin{longtable}{@{\extracolsep{\fill}}ll}
December 2021 & \parbox[t]{0.85\textwidth}{%
\textbf{\href{https://www.researchgate.net/lab/Holger-Krefts-lab-Holger-Kreft}{Holger Kreft's regular lab seminar} on 'Navigating the landscape of taxonomic harmonization'}\\[-0.1cm]{\footnotesize University of Göttingen}}\\[0.4cm]
May 2022 & \parbox[t]{0.85\textwidth}{%
\textbf{Invited talk at the 3rd meeting of \href{https://d2kab.mystrikingly.com/}{Data to Knowledge in Agronomy and Biodiversity (D2KAB)} on 'Taxonomic Databases of Plants and Animals'}\\[-0.1cm]{\footnotesize Paris/Online}}\\[0.4cm]
\end{longtable}

\hypertarget{talks}{%
\subsection{Talks}\label{talks}}

\begin{longtable}{@{\extracolsep{\fill}}ll}
September 2022 & \parbox[t]{0.85\textwidth}{%
\textbf{A barrier to global plant invasion ecology: gaps in trait availability for alien species}\\[-0.1cm]{\footnotesize Neobiota 2022, Tartu, Estonia}}\\[0.4cm]
July 2022 & \parbox[t]{0.85\textwidth}{%
\textbf{A barrier to global plant invasion ecology: gaps in trait availability for alien species}\\[-0.1cm]{\footnotesize BES Macro 2022, Online}}\\[0.4cm]
April 2022 & \parbox[t]{0.85\textwidth}{%
\textbf{A barrier to global plant invasion ecology: gaps in trait availability for alien species}\\[-0.1cm]{\footnotesize iDiv Conference 2022, Leipzig, Germany}}\\[0.4cm]
August 2021 & \parbox[t]{0.85\textwidth}{%
\textbf{Navigating the landscape of taxonomic harmonization: data, tools, and best practices}\\[-0.1cm]{\footnotesize GfÖ conference (German Speaking Ecological Society Meeting) 2021}}\\[0.4cm]
July 2021 & \parbox[t]{0.85\textwidth}{%
\textbf{Navigating the landscape of taxonomic harmonization: data, tools, and best practices}\\[-0.1cm]{\footnotesize BES Macro 2021, Online}}\\[0.4cm]
October 2018 & \parbox[t]{0.85\textwidth}{%
\textbf{Functional rarity of coral reef fishes at the global scale: Hotspots and challenges for conservation}\\[-0.1cm]{\footnotesize French Ecological Society Meeting (SFÉcologie) 2018, Rennes, France}}\\[0.4cm]
July 2018 & \parbox[t]{0.85\textwidth}{%
\textbf{Predicting Species Richness with unicorns or why should we discuss the use of thresholds?}\\[-0.1cm]{\footnotesize BES Macro (British Ecological Society Macroecology Special Interest Group Meeting) 2018, Saint-Andrews, Scotland}}\\[0.4cm]
February 2017 & \parbox[t]{0.85\textwidth}{%
\textbf{Functional rarity of coral reef fishes across space (Best Presentation Award)}\\[-0.1cm]{\footnotesize Young Natural History Scientists’ Meeting, Paris, France}}\\[0.4cm]
September 2016 & \parbox[t]{0.85\textwidth}{%
\textbf{ A case study of Functional Rarity: worldwide coral reef fishes}\\[-0.1cm]{\footnotesize EcoSummit 2016, Montpellier, France}}\\[0.4cm]
\end{longtable}

\hypertarget{blogposts}{%
\subsection{Blogposts}\label{blogposts}}

\hypertarget{bibliography}{}
\leavevmode\vadjust pre{\hypertarget{ref-Grenie_How_2022}{}}%
1. \textbf{Grenié, M.} (2022, December 6). How to Save Ggplot2 Plots in
a targets Workflow? Retrieved February 3, 2023, from
\url{https://ropensci.org/blog/2022/12/06/save-ggplot2-targets/}

\leavevmode\vadjust pre{\hypertarget{ref-Salmon_Why_2022}{}}%
2. Salmon, M., \textbf{Grenié, M.}, \& Gruson, H. (2022, June 16). Why
You Should (or Shouldn't) Build an API Client. Retrieved February 16,
2023, from
\url{https://ropensci.org/blog/2022/06/16/publicize-api-client-yes-no/}

\leavevmode\vadjust pre{\hypertarget{ref-Grenie_Best_2022}{}}%
3. \textbf{Grenié, M.}, Berti, E., Carvajal-Quintero, J., Dädlow, G. M.
L., Sagouis, A., \& Winter, M. (2022, March 2). Best practices for
taxonomic harmonization, an overlooked yet crucial step in biodiversity
analyses. Retrieved May 27, 2022, from
\url{https://methodsblog.com/2022/03/02/best-practices-for-taxonomic-harmonization-an-overlooked-yet-crucial-step-in-biodiversity-analyses/}

\leavevmode\vadjust pre{\hypertarget{ref-Grenie_Community_2020}{}}%
4. \textbf{Grenié, M.}, \& Gruson, H. (2020, July 15). Community
Captioning of rOpenSci Community Calls. Retrieved May 27, 2022, from
\url{https://ropensci.org/blog/2020/07/15/subtitles/}

\leavevmode\vadjust pre{\hypertarget{ref-Grenie_Access_2019}{}}%
5. \textbf{Grenié, M.}, \& Gruson, H. (2019, June 4). Access Publisher
Copyright \& Self-Archiving Policies via the 'SHERPA/RoMEO' API.
Retrieved May 27, 2022, from
\url{https://ropensci.org/blog/2019/06/04/rromeo/}

\hypertarget{teaching-experience}{%
\section{Teaching Experience}\label{teaching-experience}}

I taught 100+ hours on Ecology, Biodiversity Statistics, and R
programming from 2nd year of Bachelor to PhD students (detail below).

I am a certified \href{https://carpentries.org/}{Carpentries}
instructor, focusing on \textbf{R programming} and \textbf{Data
Visualization}.

I also founded the R User Group at the Center for Functional and
Evolutionary Ecology in Montpellier. There, I organized regular 1h
meet-ups on various R topics (plotting with \texttt{ggplot2}, mapping
with \texttt{sf}, project organization with RStudio, workflow
organization with \texttt{drake}, parallelization with \texttt{future}).

\textbf{Teaching detail}

\begin{longtable}[]{@{}
  >{\raggedright\arraybackslash}p{(\columnwidth - 6\tabcolsep) * \real{0.4478}}
  >{\raggedright\arraybackslash}p{(\columnwidth - 6\tabcolsep) * \real{0.1343}}
  >{\raggedright\arraybackslash}p{(\columnwidth - 6\tabcolsep) * \real{0.2388}}
  >{\raggedright\arraybackslash}p{(\columnwidth - 6\tabcolsep) * \real{0.1791}}@{}}
\toprule\noalign{}
\begin{minipage}[b]{\linewidth}\raggedright
Course Name
\end{minipage} & \begin{minipage}[b]{\linewidth}\raggedright
Year(s)
\end{minipage} & \begin{minipage}[b]{\linewidth}\raggedright
Level
\end{minipage} & \begin{minipage}[b]{\linewidth}\raggedright
Total time
\end{minipage} \\
\midrule\noalign{}
\endhead
\bottomrule\noalign{}
\endlastfoot
Descriptive Statistics with R (UM) & 2016/2017 & Bachelor's (2nd yr) &
50h \\
Functional Biogeography with R (UM,
\href{https://github.com/Rekyt/functional_biogeo_practical}{material}) &
2017/2018 & Master's (2nd yr) & 8h \\
Guest Lecture on Biodiversity Facets (UL) & 2021/2022 & Master's (1st
yr) & 4h \\
Biodiversity Facets with R (UL,
\href{https://rekyt.github.io/biodiversity_facets_tutorial/}{material})
& 2021/2022 & Master's (1st yr) & 12h \\
Group project on Biodiversity Facets (UL) & 2021 & Master's (1st yr) &
20h \\
Introduction to git and GitHub (UL/iDiv,
\href{https://emilio-berti.github.io/idiv-git-introduction}{material}) &
2022 & Masters, PhD students, and postdocs & 10h \\
Workshop on Functional Diversity and Rarity, (@CESAB,
\href{https://frbcesab.github.io/workshop-free/}{material}) & 2022 &
Masters, PhD students, postdocs, \& reseachers & 3.5h \\
Guest Lecture on Taxonomic Harmonization
\href{https://www.nfdi4biodiversity.org/en/winterschool/}{@NFDI4Biodiversity
Winter School on Data Management in Ecology and Evolution} & 2022 &
Masters, PhD students, and postdocs & 2h \\
\end{longtable}

UM: University of Montpellier; UL: University of Leipzig

\hypertarget{skills}{%
\section{Skills}\label{skills}}

\hypertarget{languages}{%
\subsection{Languages}\label{languages}}

Level estimates are given following the
\href{https://www.coe.int/en/web/portfolio/self-assessment-grid}{Common
European Framework of Reference for Languages (CEFR)}

French {[}native speaker{]}; English {[}C2, fluent speaker{]}; Spanish
{[}B1-B2{]}; German {[}A2-B1{]}; Chinese {[}A2-B1{]}; Japanese {[}A2{]}

\hypertarget{programming-languages}{%
\subsection{Programming languages}\label{programming-languages}}

\begin{itemize}
\tightlist
\item
  \faRProject~\textbf{R} language ~\textbullet~ Best Package Development
  Practices (\texttt{testthat} unit tests, \texttt{pkgdown} website,
  Continuous Integration/Continuous Deployment, submission to CRAN)
  ~\textbullet~ data visualization (advanced \texttt{ggplot2} and
  extension packages) ~\textbullet~ spatial data manipulation
  {[}\texttt{terra} rasters, \texttt{sf} vectors, PostGIS database, and
  interactions{]}
\item
  \faDatabase~\textbf{SQL} interaction with databases ~\textbullet~
  high-performance query ~\textbullet~ interaction with \texttt{R}
  ~\textbullet~ PostgreSQL/PostGIS spatial database
\item
  \faServer~\textbf{High Performance Computing} job submission on
  SLURM/SGE ~\textbullet~ complex array jobs
\item
  \faGit~version control ~\textbullet~ git flow framework ~\textbullet~
  collaborative development through \faGithub~GitHub
\item
  Basic knowledge of \faPython~\textbf{Python} and \textbf{Julia}
\end{itemize}

\hypertarget{other}{%
\subsection{Other}\label{other}}

Digital Photography, Film Photography, Hiking

\end{document}
