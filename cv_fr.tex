%% Template for a CV
%% Author: Rob J Hyndman

\documentclass[12pt,a4paper,]{article}
\usepackage[scaled=0.86]{DejaVuSansMono}
\usepackage[sfdefault,lf,t]{carlito}

% Change color to blue
\usepackage{color,xcolor}
\definecolor{headcolor}{HTML}{990000}

\usepackage{ifxetex,ifluatex}
\usepackage{fixltx2e} % provides \textsubscript
\ifnum 0\ifxetex 1\fi\ifluatex 1\fi=0 % if pdftex
  \usepackage[T1]{fontenc}
  \usepackage[utf8]{inputenc}
\else % if luatex or xelatex
  \ifxetex
    \usepackage{mathspec}
  \else
    \usepackage{fontspec}
  \fi
  \defaultfontfeatures{Ligatures=TeX,Scale=MatchLowercase}
\fi

\usepackage[utf8]{inputenc}
\usepackage[T1]{fontenc}

% use upquote if available, for straight quotes in verbatim environments
\IfFileExists{upquote.sty}{\usepackage{upquote}}{}
% use microtype if available
\IfFileExists{microtype.sty}{%
\usepackage[]{microtype}
\UseMicrotypeSet[protrusion]{basicmath} % disable protrusion for tt fonts
}{}
\PassOptionsToPackage{hyphens}{url} % url is loaded by hyperref
\usepackage[unicode=true,hidelinks]{hyperref}
\urlstyle{same}  % don't use monospace font for urls
\usepackage{geometry}
\geometry{left=1.75cm,right=1.75cm,top=2.2cm,bottom=2cm}

\usepackage{longtable,booktabs}
% Fix footnotes in tables (requires footnote package)
\IfFileExists{footnote.sty}{\usepackage{footnote}\makesavenoteenv{long table}}{}
\IfFileExists{parskip.sty}{%
\usepackage{parskip}
}{% else
\setlength{\parindent}{0pt}
\setlength{\parskip}{6pt plus 2pt minus 1pt}
}
\setlength{\emergencystretch}{3em}  % prevent overfull lines
\providecommand{\tightlist}{%
  \setlength{\itemsep}{0pt}\setlength{\parskip}{0pt}}
\setcounter{secnumdepth}{0}

% set default figure placement to htbp
\makeatletter
\def\fps@figure{htbp}
\makeatother



\date{juin 2024}

\definecolor{headcolor}{HTML}{1E49BF}
\definecolor{linkscolor}{HTML}{026dbb}
\hypersetup{colorlinks=true, linkcolor=linkscolor, filecolor=linkscolor, urlcolor=linkscolor, urlbordercolor=linkscolor}

\usepackage{paralist,ragged2e,datetime}
\usepackage[hyphens]{url}
\usepackage{fancyhdr,enumitem,pifont}
\usepackage[compact,small,sf,bf]{titlesec}

\RaggedRight
\sloppy

% Header and footer
\pagestyle{fancy}
\makeatletter
\lhead{\sf\textcolor[gray]{0.4}{Curriculum Vitae: \@name}}
\rhead{\sf\textcolor[gray]{0.4}{\thepage}}
\cfoot{}
\def\headrule{{\color[gray]{0.4}\hrule\@height\headrulewidth\@width\headwidth \vskip-\headrulewidth}}
\makeatother

% Header box
\usepackage{tabularx}

\makeatletter
\def\name#1{\def\@name{#1}}
\def\info#1{\def\@info{#1}}
\makeatother
\newcommand{\shadebox}[3][.9]{\fcolorbox[gray]{0}{#1}{\parbox{#2}{#3}}}

\usepackage{calc}
\newlength{\headerboxwidth}
\setlength{\headerboxwidth}{\textwidth}
%\addtolength{\headerboxwidth}{0.2cm}
\makeatletter
\def\maketitle{
\thispagestyle{plain}
\vspace*{-1.4cm}
\shadebox[0.9]{\headerboxwidth}{\sf\color{headcolor}\hfil
\hbox to 0.98\textwidth{\begin{tabular}{l}
\\[-0.3cm]
\LARGE\textbf{\@name}
\\[0.1cm]\large Maître de conférences\\[0.6cm]
\normalsize\textbf{Curriculum Vitae}\\
\normalsize juin 2024
\end{tabular}
\hfill\hbox{\fontsize{9}{12}\sf
\begin{tabular}{@{}rl@{}}
\@info
\end{tabular}}}\hfil
}
\vspace*{0.2cm}}
\makeatother

% Section headings
\titlelabel{}
\titlespacing{\section}{0pt}{1.5ex}{0.5ex}
\titleformat*{\section}{\color{headcolor}\large\sf\bfseries}
\titleformat*{\subsection}{\color{headcolor}\sf\bfseries}
\titlespacing{\subsection}{0pt}{1ex}{0.5ex}

% Miscellaneous dimensions
\setlength{\parskip}{0ex}
\setlength{\parindent}{0em}
\setlength{\headheight}{15pt}
\setlength{\tabcolsep}{0.15cm}
\clubpenalty = 10000
\widowpenalty = 10000
\setlist{itemsep=1pt}
\setdescription{labelwidth=1.2cm,leftmargin=1.5cm,labelindent=1.5cm,font=\rm}

% Make nicer bullets
\renewcommand{\labelitemi}{\ding{228}}

\usepackage{booktabs,fontawesome5}
%\usepackage[t1,scale=0.86]{sourcecodepro}

\name{Matthias Grenié}
\def\imagetop#1{\vtop{\null\hbox{#1}}}
\info{%
\raisebox{-0.05cm}{\imagetop{\faIcon{map-marker-alt}}} &  \imagetop{\begin{tabular}{@{}l@{}}Université
Grenoble Alpes (UGA), Laboratoire d'Écologie Alpine
(LECA)\end{tabular}}\\ %
\faIcon{home} & \href{http://rekyt.github.io}{rekyt.github.io}\\% %
%
\faIcon{phone-alt} & +33 4 76 51 41 15\\%
\faIcon{envelope} & \href{mailto:matthias.grenie@idiv.de}{\nolinkurl{matthias.grenie@idiv.de}}\\%
\faIcon{twitter} & \href{https://twitter.com/LeNematode}{@LeNematode}\\%
\faIcon{github} & \href{https://github.com/Rekyt}{Rekyt}\\%
%
\faIcon{google} & \href{https://scholar.google.com/citations?user=fZ1\_d7QAAAAJ}{fZ1\_d7QAAAAJ}\\%
\faIcon{researchgate} & \href{https://www.researchgate.net/profile/Matthias-Grenie}{Matthias-Grenie}\\%
\faIcon{orcid} & \href{https://orcid.org/0000-0002-4659-7522}{0000-0002-4659-7522}\\%
}


%\usepackage{inconsolata}


\setlength\LTleft{0pt}
\setlength\LTright{0pt}

% Pandoc CSL macros
% definitions for citeproc citations
\NewDocumentCommand\citeproctext{}{}
\NewDocumentCommand\citeproc{mm}{%
  \begingroup\def\citeproctext{#2}\cite{#1}\endgroup}
\makeatletter
 % allow citations to break across lines
 \let\@cite@ofmt\@firstofone
 % avoid brackets around text for \cite:
 \def\@biblabel#1{}
 \def\@cite#1#2{{#1\if@tempswa , #2\fi}}
\makeatother
\newlength{\cslhangindent}
\setlength{\cslhangindent}{1em}
\newlength{\csllabelwidth}
\setlength{\csllabelwidth}{3em}
\newenvironment{CSLReferences}[2] % #1 hanging-indent, #2 entry-spacing
 {\begin{list}{}{%
  \setlength{\itemindent}{0pt}
  \setlength{\leftmargin}{0pt}
  \setlength{\parsep}{0pt}
  % turn on hanging indent if param 1 is 1
  \ifodd #1
   \setlength{\leftmargin}{\cslhangindent}
   \setlength{\itemindent}{-1\cslhangindent}
  \fi
  % set entry spacing
  \setlength{\itemsep}{#2\baselineskip}}}
 {\end{list}}
\usepackage{calc}
\newcommand{\CSLBlock}[1]{\hfill\break\parbox[t]{\linewidth}{\strut\ignorespaces#1\strut}}
\newcommand{\CSLLeftMargin}[1]{\parbox[t]{\csllabelwidth}{\strut#1\strut}}
\newcommand{\CSLRightInline}[1]{\parbox[t]{\linewidth - \csllabelwidth}{\strut#1\strut}}
\newcommand{\CSLIndent}[1]{\hspace{\cslhangindent}#1}


\def\endfirstpage{\newpage}

\begin{document}
\maketitle


\section{Cursus Universitaire}\label{cursus-universitaire}

\begin{longtable}{@{\extracolsep{\fill}}ll}
2016-2020 & \parbox[t]{0.85\textwidth}{%
\textbf{Doctorat en Écologie}\hfill{\footnotesize École Doctorale 584 GAIA, Univ. Montpellier}\newline
  Montpellier, France\par%
  \vspace{0.1cm}\begin{minipage}{0.7\textwidth}%
\begin{itemize}%
\item Directeurs: \textbf{Prof. Françoiz Munoz} et \textbf{Dr. Cyrille Violle}\\~Titre: « En dehors de la norme: déviation de l'optimalité écologique et rareté fonctionnelle »%
\end{itemize}%
\end{minipage}%
\vspace{\parsep}}\\
2013-2015 & \parbox[t]{0.85\textwidth}{%
\textbf{Master Biosciences}\hfill{\footnotesize École Normale Supérieure de Lyon, Univ. Claude Bernard Lyon 1}\newline
  Lyon, France\par%
  \vspace{0.1cm}\begin{minipage}{0.7\textwidth}%
\begin{itemize}%
\item Normalien élève%
\end{itemize}%
\end{minipage}%
\vspace{\parsep}}\\
2012-2013 & \parbox[t]{0.85\textwidth}{%
\textbf{Licence Biosciences}\hfill{\footnotesize École Normale Supérieure de Lyon, Univ. Claude Bernard Lyon 1}\newline
  Lyon, France\par%
  \vspace{0.1cm}\begin{minipage}{0.7\textwidth}%
\begin{itemize}%
\item Normalien élève%
\end{itemize}%
\end{minipage}%
\vspace{\parsep}}\\
2010-2012 & \parbox[t]{0.85\textwidth}{%
\textbf{Classe Préparatoire BCPST}\hfill{\footnotesize Lycée Saint-Louis}\newline
  Paris, France\par%
  \vspace{0.1cm}\begin{minipage}{0.7\textwidth}%
\begin{itemize}%
\item Admis au concours Agronomie et ENS Lyon%
\end{itemize}%
\end{minipage}%
\vspace{\parsep}}\\
\end{longtable}

\section{Parcours Professionnel
Académique}\label{parcours-professionnel-acaduxe9mique}

\begin{longtable}{@{\extracolsep{\fill}}ll}
2023 - \textbf{présent} & \parbox[t]{0.85\textwidth}{%
\textbf{Maître de Conférences}\hfill{\footnotesize Université Grenoble Alpes (UGA), Laboratoire d'Écologie Alpine (LECA)}\newline
  Grenoble, France\par%
  \vspace{0.1cm}\begin{minipage}{0.7\textwidth}%
\begin{itemize}%
\item associé à la faculté de Pharmacie, enseignant les statistiques appliquées%
\end{itemize}%
\end{minipage}%
\vspace{\parsep}}\\
2020 - 2023 & \parbox[t]{0.85\textwidth}{%
\textbf{Chercheur Postdoctoral}\hfill{\footnotesize German Center for Integrative Biodiversity Research (iDiv) / Leipzig University}\newline
  Leipzig, Allemagne\par%
  \vspace{0.1cm}\begin{minipage}{0.7\textwidth}%
\begin{itemize}%
\item Supervisé par \textbf{Dr. Marten Winter} sur la biogéographie fonctionnelle des plantes introduites%
\end{itemize}%
\end{minipage}%
\vspace{\parsep}}\\
2016-2020 & \parbox[t]{0.85\textwidth}{%
\textbf{Doctorant}\hfill{\footnotesize Université de Montpellier}\newline
  Montpellier, France\par%
  \vspace{0.1cm}\begin{minipage}{0.7\textwidth}%
\begin{itemize}%
\item Directeurs: \textbf{Pr. François Munoz} et \textbf{Dr. Cyrille Violle}. Contrat doctoral spécifique normalien. Sur la rareté fonctionnelle et l'écologie fonctionnelle des communautés%
\end{itemize}%
\end{minipage}%
\vspace{\parsep}}\\
\end{longtable}

\newpage

\section{Encadrement Scientifique}\label{encadrement-scientifique}

\begin{longtable}{@{\extracolsep{\fill}}ll}
sept. 2023-\textbf{présent} & \parbox[t]{0.85\textwidth}{%
\textbf{Membre du comité de thèse de \href{https://orcid.org/0000-0001-6931-2879}{Marianne Tzivanopoulos}}\\[-0.1cm]{\footnotesize }}\\[0.4cm]
2022-2023 & \parbox[t]{0.85\textwidth}{%
\textbf{Membre du comité de thèse de \href{https://www.idiv.de/en/profile/1248.html}{Rachel Souza Fereira}}\\[-0.1cm]{\footnotesize }}\\[0.4cm]
avr. 2019 - juin 2019 & \parbox[t]{0.85\textwidth}{%
\textbf{Co-encadrement stage de M1}\\[-0.1cm]{\footnotesize Nathan Mazet, stage sur les stratégies alimentaires des oiseaux à l'échelle globale, encadrant principal: Pr. Jean-Yves Barnagaud}}\\[0.4cm]
avr. 2018 - mai 2018 & \parbox[t]{0.85\textwidth}{%
\textbf{Encadrement stage de M1}\\[-0.1cm]{\footnotesize Charlotte Guérineau, stage sur la rareté fonctionnelle de différents groupes taxonomiques en France}}\\[0.4cm]
\end{longtable}

\section{Activités d'organisation et responsabilités
collectives}\label{activituxe9s-dorganisation-et-responsabilituxe9s-collectives}

\begin{longtable}{@{\extracolsep{\fill}}ll}
2022-\textbf{présent} & \parbox[t]{0.85\textwidth}{%
\textbf{Relecteur de logiciel scientifique pour rOpenSci}\\[-0.1cm]{\footnotesize Pour le paquet \href{https://github.com/ropensci/software-review/issues/505}{\texttt{npi}}}}\\[0.4cm]
2022-2023 & \parbox[t]{0.85\textwidth}{%
\textbf{Co-fondateur et animateur de \textit{iCode}}\\[-0.1cm]{\footnotesize Groupe d'animation et d'entraide (forum d'échanges, séminaires) sur R et la programmation à iDiv}}\\[0.4cm]
2022-2023 & \parbox[t]{0.85\textwidth}{%
\textbf{Représentant des post-doctorant$\cdotp$es (130 personnes)}\\[-0.1cm]{\footnotesize Au \href{https://www.idiv.de/en/council.html}{\textbf{conseil de laboratoire d'iDiv}}, équivalent du conseil d'UMR}}\\[0.4cm]
2021-2023 & \parbox[t]{0.85\textwidth}{%
\textbf{Organisateur des séminaires de l'équipe sDiv}\\[-0.1cm]{\footnotesize toutes les deux semaines}}\\[0.4cm]
2018-2020 & \parbox[t]{0.85\textwidth}{%
\textbf{Représentant des Doctorant$\cdotp$es (70 personnes)}\\[-0.1cm]{\footnotesize Conseil d'UMR du CEFE}}\\[0.4cm]
2018 & \parbox[t]{0.85\textwidth}{%
\textbf{Co-organisateur d'un atelier sur les pratiques d'\textit{Open Access}}\\[-0.1cm]{\footnotesize Un événement d'une demi-journée pour apprendre à rendre nos articles accessibles, pour \textasciitilde{}50 personnes au CEFE à Montpellier}}\\[0.4cm]
2017-\textbf{présent} & \parbox[t]{0.85\textwidth}{%
\textbf{Relecteur de journaux scientifiques (\href{https://publons.com/wos-op/researcher/1466931/matthias-grenie}{voir le profil Publons})}\\[-0.1cm]{\footnotesize Pour \textit{Ecology Letters}, \textit{Functional Ecology}, \textit{Biological Reviews}, \textit{PeerJ}, \textit{Biological Conservation}, \textit{New Phytologist}, \textit{Journal of Applied Ecology}, et \textit{Journal of Biogeography}}}\\[0.4cm]
2017 & \parbox[t]{0.85\textwidth}{%
\textbf{Co-organisateur de la conférence Jeune Chercheur$\cdotp$euses \href{https://web.archive.org/web/20180823012054/http://www.mee.univ-montp2.fr/editions-precedentes/edition-2017/organisateurs-2017/}{'Models in Ecology and Evolution'}}\\[-0.1cm]{\footnotesize Une conférence annuelle pour jeunes chercheur$\cdotp$euses avec un public de \textasciitilde{}100 personnes du Master au Post-doctorat}}\\[0.4cm]
2016-2020 & \parbox[t]{0.85\textwidth}{%
\textbf{Fondateur et animateur du groupe d'utilisateur$\cdotp$rices de R au CEFE}\\[-0.1cm]{\footnotesize Animation mensuelle pour les personnes utilisant R au CEFE, incluant présentations et ateliers pratiques}}\\[0.4cm]
\end{longtable}

\section{Participation à des projets
scientifiques}\label{participation-uxe0-des-projets-scientifiques}

\begin{longtable}{@{\extracolsep{\fill}}ll}
2024-\textbf{présent} & \parbox[t]{0.85\textwidth}{%
\textbf{\href{https://www.obsgession.eu}{Projet Européen OBSGESSION}}\hfill{\footnotesize Europe}\newline
  \empty%
  \vspace{0.1cm}\begin{minipage}{0.7\textwidth}%
\begin{itemize}%
\item Projet européen regroupant 11 partenaires institutionnels différents pour améliorer la biologie de la conservation en mobilisant notamment la télédétection et l'inférence causale.%
\end{itemize}%
\end{minipage}%
\vspace{\parsep}}\\
2024-\textbf{présent} & \parbox[t]{0.85\textwidth}{%
\textbf{\href{https://www.idiv.de/en/sfragment.html}{Groupe sDiv sFragment}}\hfill{\footnotesize sDiv}\newline
  \empty%
  \vspace{0.1cm}\begin{minipage}{0.7\textwidth}%
\begin{itemize}%
\item groupe de synthèse 'jeune chercheurs' sDiv pour tester relation entre fragmentation et diversité fonctionnelle des oiseaux et des arbres. Membre en tant qu'expert sur les traits et la diversité fonctionnelle.%
\end{itemize}%
\end{minipage}%
\vspace{\parsep}}\\
2022-\textbf{présent} & \parbox[t]{0.85\textwidth}{%
\textbf{Groupe \href{https://www.fondationbiodiversite.fr/la-frb-en-action/programmes-et-projets/le-cesab/impacts/}{IMPACTS}}\hfill{\footnotesize Centre de Synthèse et d'Analyse sur la Biodiversité (CESAB)}\newline
  \empty%
  \vspace{0.1cm}\begin{minipage}{0.7\textwidth}%
\begin{itemize}%
\item Groupe de synthèse sur le suivi temporel de la biodiversité terrestre en France\break Membre du groupe, en charge de la gestion de données\break Première réunion en présentiel : mars 2023%
\end{itemize}%
\end{minipage}%
\vspace{\parsep}}\\
2021-\textbf{présent} & \parbox[t]{0.85\textwidth}{%
\textbf{Groupe \href{https://glonaf.org/}{Global Naturalized Alien Flora}}\hfill{\footnotesize GloNAF}\newline
  \empty%
  \vspace{0.1cm}\begin{minipage}{0.7\textwidth}%
\begin{itemize}%
\item Groupe européen sur les invasions végétales \textit{via} la base GloNAF \break Contributeur aux réunions bisannuelles du groupe \break Premier auteur d'un manuscrit pour le groupe -- Co-auteur d'un manuscrit%
\end{itemize}%
\end{minipage}%
\vspace{\parsep}}\\
2017-\textbf{présent} & \parbox[t]{0.85\textwidth}{%
\textbf{\href{https://www.fondationbiodiversite.fr/la-frb-en-action/programmes-et-projets/le-cesab/free-2/}{Groupe FREE2}}\hfill{\footnotesize CESAB}\newline
  \empty%
  \vspace{0.1cm}\begin{minipage}{0.7\textwidth}%
\begin{itemize}%
\item Groupe de synthèse sur la rareté fonctionnelle en écologie and évolution \break Membre co-fondateur du groupe \break Premier auteur de 3 articles liés au groupe -- Co-auteur de 4 articles%
\end{itemize}%
\end{minipage}%
\vspace{\parsep}}\\
\end{longtable}

\section{Bourses et Financements}\label{bourses-et-financements}

\begin{longtable}{@{\extracolsep{\fill}}ll}
2023-\textbf{présent} & \parbox[t]{0.85\textwidth}{%
\textbf{Financement}\hfill{\footnotesize Flexpool (interne iDiv), poste de post-doctorante}\newline
  \empty%
  \vspace{0.1cm}\begin{minipage}{0.7\textwidth}%
\begin{itemize}%
\item Collaborateur du projet mené par Dr. Sonja Knapp et de la postdoctorante Dr. Laura Mendéz-Cuellar sur les déterminants des extinctions régionales d'espèces%
\end{itemize}%
\end{minipage}%
\vspace{\parsep}}\\
2023-\textbf{présent} & \parbox[t]{0.85\textwidth}{%
\textbf{Financement}\hfill{\footnotesize Flexpool (interne iDiv) 10k€}\newline
  \empty%
  \vspace{0.1cm}\begin{minipage}{0.7\textwidth}%
\begin{itemize}%
\item Co-rédacteur et collaborateur du projet mené par Dr. Qiang Yang sur l'impact des espèces de plantes naturalisées sur les réseaux plantes-pollinisateurs le long de gradients altitudinaux%
\end{itemize}%
\end{minipage}%
\vspace{\parsep}}\\
2022-\textbf{présent} & \parbox[t]{0.85\textwidth}{%
\textbf{Financement}\hfill{\footnotesize Flexpool (interne iDiv) 10k€}\newline
  \empty%
  \vspace{0.1cm}\begin{minipage}{0.7\textwidth}%
\begin{itemize}%
\item Co-rédacteur et collaborateur du projet mené par Dr. Bettina Ohse, sur le lien entre traits fonctionnels et taux démographiques des arbres%
\end{itemize}%
\end{minipage}%
\vspace{\parsep}}\\
Sept. 2016 & \parbox[t]{0.85\textwidth}{%
\textbf{Bourse doctorale}\hfill{\footnotesize École Normale Supérieure de Lyon}\newline
  \empty%
  \vspace{0.1cm}\begin{minipage}{0.7\textwidth}%
\begin{itemize}%
\item Contrat Doctoral Spécifique Normalien%
\end{itemize}%
\end{minipage}%
\vspace{\parsep}}\\
\end{longtable}

\section{Contributions scientifiques}\label{contributions-scientifiques}

\faFile*~16 publications (5 en premier auteur). \faQuoteLeft~764
citations (Google Scholar). \faHSquare~h-index 11 (Google Scholar).
\faRProject~6 paquets R

\subsection{Publications}\label{publications}

\subsubsection{Acceptée}\label{acceptuxe9e}

\phantomsection\label{refs-b7e7f3d247ef10bbb35423e631999e1e}
\begin{CSLReferences}{1}{1}
\bibitem[\citeproctext]{ref-Pili_forecasting_accepted}
1. Pili, A., Measey, J., Farquhar, J., \ldots, \textbf{Grenié, M.},
\ldots, Zurell, D., Courchamp, F., \& Chapple, D. (2024). Forecasting
potential invaders to prevent future biological invasions worldwide.
\emph{Accepted in Global Change Biology}.

\end{CSLReferences}

\subsubsection{Publiées}\label{publiuxe9es}

\phantomsection\label{refs-86c4a80be2a3d18425d3a7c99120529a}
\begin{CSLReferences}{1}{0.5}
\bibitem[\citeproctext]{ref-Cutts_Links_2023}
1. Cutts, V., Hanz, D. M., Barajas-Barbosa, M. P., Schrodt, F.,
Steinbauer, M. J., Beierkuhnlein, C., Denelle, P., Fernández-Palacios,
J. M., Gaüzère, P., \textbf{Grenié, M.}, Irl, S. D. H., \ldots{} Algar,
A. C. (2023). Links to rare climates do not translate into distinct
traits for island endemics. \emph{Ecology Letters}, \emph{26}(4),
504--515. \url{https://doi.org/10.1111/ele.14169}

\bibitem[\citeproctext]{ref-Grenie_Fundiversity_2023}
2. \textbf{Grenié, M.}, \& Gruson, H. (2023). Fundiversity: A modular R
package to compute functional diversity indices. \emph{Ecography},
\emph{2023}(3), e06585. \url{https://doi.org/10.1111/ecog.06585}

\bibitem[\citeproctext]{ref-Munoz_Ecological_2023}
3. Munoz, F., Klausmeier, C. A., Gaüzère, P., Kandlikar, G., \ldots,
\textbf{Grenié, M.}, \ldots, Violle, C., \& Kraft, N. J. B. (2023). The
ecological causes of functional distinctiveness in communities.
\emph{Ecology Letters}, \emph{26}(8), 1452--1465.
\url{https://doi.org/10.1111/ele.14265}

\bibitem[\citeproctext]{ref-Gauzere_functional_2022}
4. Gaüzère, P., Denelle, P., Fournier, B., \textbf{Grenié, M.},
Delalandre, L., Münkemüller, T., Munoz, F., Violle, C., \& Thuiller, W.
(2022). The functional trait distinctiveness of plant species is scale
dependent. \emph{Ecography}, \emph{in press}.

\bibitem[\citeproctext]{ref-Grenie_Harmonizing_2022}
5. \textbf{Grenié, M.}, Berti, E., Carvajal-Quintero, J., Dädlow, G. M.
L., Sagouis, A., \& Winter, M. (2022). Harmonizing taxon names in
biodiversity data: A review of tools, databases and best practices.
\emph{Methods in Ecology and Evolution}, 2041--210X.13802.
\url{https://doi.org/10.1111/2041-210X.13802}

\bibitem[\citeproctext]{ref-Jurburg_community_2022}
6. Jurburg, S. D., Buscot, F., Chatzinotas, A., Chaudhari, N. M., Clark,
A. T., Garbowski, M., \textbf{Grenié, M.}, Hom, E. F. Y., Karakoç, C.,
Marr, S., Neumann, S., \ldots{} Heintz-Buschart, A. (2022). The
community ecology perspective of omics data. \emph{Microbiome},
\emph{10}(1), 225. \url{https://doi.org/10.1186/s40168-022-01423-8}

\bibitem[\citeproctext]{ref-Mouillot_dimensionality_2021}
7. Mouillot, D., Loiseau, N., \textbf{Grenié, M.}, Algar, A. C.,
Allegra, M., Cadotte, M. W., Casajus, N., Denelle, P., Guéguen, M.,
Maire, A., Maitner, B., \ldots{} Auber, A. (2021). The dimensionality
and structure of species trait spaces. \emph{Ecology Letters},
\emph{24}(9), 1988--2009. \url{https://doi.org/10.1111/ele.13778}

\bibitem[\citeproctext]{ref-Grenie_prediction_2020}
8. \textbf{Grenié, M.}, Violle, C., \& Munoz, F. (2020). Is prediction
of species richness from stacked species distribution models biased by
habitat saturation? \emph{Ecological Indicators}, \emph{111}, 105970.
\url{https://doi.org/10.1016/j.ecolind.2019.105970}

\bibitem[\citeproctext]{ref-Loiseau_Global_2020}
9. Loiseau, N., Mouquet, N., Casajus, N., \textbf{Grenié, M.}, Guéguen,
M., Maitner, B., Mouillot, D., Ostling, A., Renaud, J., Tucker, C.,
Velez, L., \ldots{} Violle, C. (2020). Global distribution and
conservation status of ecologically rare mammal and bird species.
\emph{Nature Communications}, \emph{11}(1, 1), 5071.
\url{https://doi.org/10.1038/s41467-020-18779-w}

\bibitem[\citeproctext]{ref-Barnagaud_Functional_2019}
10. Barnagaud, J.-Y., Mazet, N., Munoz, F., \textbf{Grenié, M.},
Denelle, P., Sobral, M., Kissling, W. D., Sekercioglu, Ç. H., \& Violle,
C. (2019). Functional biogeography of dietary strategies in birds.
\emph{Global Ecology and Biogeography}, \emph{28}(7), 1004--1017.
\url{https://doi.org/10.1111/geb.12910}

\bibitem[\citeproctext]{ref-Joffard_Effect_2019}
11. Joffard, N., Massol, F., \textbf{Grenié, M.}, Montgelard, C., \&
Schatz, B. (2019). Effect of pollination strategy, phylogeny and
distribution on pollination niches of Euro-Mediterranean orchids.
\emph{Journal of Ecology}, \emph{107}(1), 478--490.
\url{https://doi.org/10.1111/1365-2745.13013}

\bibitem[\citeproctext]{ref-Grenie_Functional_2018}
12. \textbf{Grenié, M.}, Mouillot, D., Villéger, S., Denelle, P.,
Tucker, C. M., Munoz, F., \& Violle, C. (2018). Functional rarity of
coral reef fishes at the global scale: Hotspots and challenges for
conservation. \emph{Biological Conservation}, \emph{226}, 288--299.
\url{https://doi.org/10.1016/j.biocon.2018.08.011}

\bibitem[\citeproctext]{ref-Munoz_ecolottery_2018}
13. Munoz, F., \textbf{Grenié, M.}, Denelle, P., Taudière, A., Laroche,
F., Tucker, C., \& Violle, C. (2018). Ecolottery: Simulating and
assessing community assembly with environmental filtering and neutral
dynamics in R. \emph{Methods in Ecology and Evolution}, \emph{9}(3),
693--703. \url{https://doi.org/10.1111/2041-210X.12918}

\bibitem[\citeproctext]{ref-Grenie_funrar_2017}
14. \textbf{Grenié, M.}, Denelle, P., Tucker, C. M., Munoz, F., \&
Violle, C. (2017). Funrar: An R package to characterize functional
rarity. \emph{Diversity and Distributions}, \emph{23}(12), 1365--1371.
\url{https://doi.org/10.1111/ddi.12629}

\bibitem[\citeproctext]{ref-Violle_Common_2017}
15. Violle, C., Thuiller, W., Mouquet, N., Munoz, F., Kraft, N. J. B.,
Cadotte, M. W., Livingstone, S. W., \textbf{Grenié, M.}, \& Mouillot, D.
(2017). A Common Toolbox to Understand, Monitor or Manage Rarity? A
Response to Carmona et al. \emph{Trends in Ecology \& Evolution},
\emph{32}(12), 891--893.
\url{https://doi.org/10.1016/j.tree.2017.10.001}

\end{CSLReferences}

\subsection{Paquets R}\label{paquets-r}

\begin{itemize}
\item
  \textbf{\texttt{funrar}} (créateur principal et mainteneur)
  \hfill\break calcule des indices de rareté fonctionnelle, central pour
  les travaux du groupe CESAB FREE
  \hfill\break ~\href{https://doi.org/10.1111/ddi.12629}{\faFile*~
  article publié} --
  \href{https://cran.r-project.org/package=funrar}{\faRProject~CRAN} --
  \href{https://github.com/Rekyt/funrar}{\faGithub~GitHub}
\item
  \textbf{\texttt{ecolottery}} (co-créateur) \hfill\break simule de
  manière efficace des processus d'assemblages de communautés en
  intégrant à la fois des processus neutre et de niches
  \hfill\break ~\href{https://doi.org/10.1111/2041-210X.12918}{\faFile*~article
  publié} --
  \href{https://cran.r-project.org/package=ecolottery}{\faRProject~CRAN}
  -- \href{https://github.com/frmunoz/ecolottery}{\faGithub~GitHub}
\item
  \textbf{\texttt{fundiversity}} (co-créateur et mainteneur)
  \hfill\break calcul des indices de diversité fonctionnelle de manière
  efficace, modulaire, en intégrant des fonctionnalités modernes
  récentes (parallélisation, mémoisation)
  \hfill\break ~\href{https://doi.org/10.1111/ecog.06585}{\faFile*~article
  publié} --
  \href{https://cran.r-project.org/package=fundiversity}{\faRProject~CRAN}
  -- \href{https://github.com/bisaloo/fundiversity}{\faGithub~GitHub}
\item
  \textbf{\texttt{funbiogeo}} (co-créateur et mainteneur)
  \hfill\break facilite les analyses en biogéographique fonctionnelle,
  dévelopé dans le contexte du groupe CESAB FREE
  \hfill\break ~\href{https://github.com/FRBCesab/funbiogeo}{\faGithub~GitHub}
\item
  \textbf{\texttt{rtaxref}} (créateur et maintenuer) \hfill\break accède
  aux données de l'API du Référentiel Taxonomique Français (TAXREF)
  \hfill\break ~\href{https://github.com/Rekyt/rtaxref}{\faGithub~GitHub}
\item
  Contributions significatives à d'autres paquets R:
  \href{https://cran.r-project.org/package=taxize}{\textbf{\texttt{taxize}}}
  (extraction et standardisation de données taxonomiques),
  \href{https://cran.r-project.org/package=ggfortify}{\textbf{\texttt{ggfortify}}}
  (simplification et extraction automatique de graphiques), et
  \href{https://cran.r-project.org/package=traitdataform}{\textbf{\texttt{traitdataform}}}
  (aide à l'écriture de métadonnées sur les traits fonctionnels)
\end{itemize}

\subsection{Actes de Conférences}\label{actes-de-confuxe9rences}

\phantomsection\label{refs-19236448c98ed3c3f0354a7b67ef58d9}
\begin{CSLReferences}{1}{1}
\bibitem[\citeproctext]{ref-Grenie_Matching_2021}
1. \textbf{Grenié, M.}, Berti, E., Carvajal-Quintero, J., Winter, M., \&
Sagouis, A. (2021). Matching Species Names Across Biodiversity
Databases: Sources, tools, pitfalls and best practices for taxonomic
harmonization. \emph{Biodiversity Information Science and Standards},
\emph{5}, e75359. \url{https://doi.org/10.3897/biss.5.75359}

\end{CSLReferences}

\subsection{Présentations Invitées}\label{pruxe9sentations-invituxe9es}

\begin{longtable}{@{\extracolsep{\fill}}ll}
déc. 2021 & \parbox[t]{0.85\textwidth}{%
\textbf{Séminaire de l'équipe de \href{https://www.researchgate.net/lab/Holger-Krefts-lab-Holger-Kreft}{Holger Kreft} sur 'Navigating the landscape of taxonomic harmonization'}\\[-0.1cm]{\footnotesize Université de Göttingen, Allemagne}}\\[0.4cm]
mai 2022 & \parbox[t]{0.85\textwidth}{%
\textbf{Présentation invitée à la 3ème réunion de \href{https://d2kab.mystrikingly.com/}{Data to Knowledge in Agronomy and Biodiversity (D2KAB)} sur 'Taxonomic Databases of Plants and Animals'}\\[-0.1cm]{\footnotesize Paris/en visioconférence}}\\[0.4cm]
\end{longtable}

\subsection{Présentations Orales}\label{pruxe9sentations-orales}

\begin{longtable}{@{\extracolsep{\fill}}ll}
sept 2022 & \parbox[t]{0.85\textwidth}{%
\textbf{A barrier to global plant invasion ecology: gaps in trait availability for alien species}\\[-0.1cm]{\footnotesize Neobiota 2022, Tartu, Estonia}}\\[0.4cm]
juil 2022 & \parbox[t]{0.85\textwidth}{%
\textbf{A barrier to global plant invasion ecology: gaps in trait availability for alien species}\\[-0.1cm]{\footnotesize BES Macro 2022, Online}}\\[0.4cm]
avr 2022 & \parbox[t]{0.85\textwidth}{%
\textbf{A barrier to global plant invasion ecology: gaps in trait availability for alien species}\\[-0.1cm]{\footnotesize iDiv Conference 2022, Leipzig, Germany}}\\[0.4cm]
août 2021 & \parbox[t]{0.85\textwidth}{%
\textbf{Navigating the landscape of taxonomic harmonization: data, tools, and best practices}\\[-0.1cm]{\footnotesize GfÖ conference (German Speaking Ecological Society Meeting) 2021}}\\[0.4cm]
juil 2021 & \parbox[t]{0.85\textwidth}{%
\textbf{Navigating the landscape of taxonomic harmonization: data, tools, and best practices}\\[-0.1cm]{\footnotesize BES Macro 2021, Online}}\\[0.4cm]
oct 2018 & \parbox[t]{0.85\textwidth}{%
\textbf{Functional rarity of coral reef fishes at the global scale: Hotspots and challenges for conservation}\\[-0.1cm]{\footnotesize French Ecological Society Meeting (SFÉcologie) 2018, Rennes, France}}\\[0.4cm]
juil 2018 & \parbox[t]{0.85\textwidth}{%
\textbf{Predicting Species Richness with unicorns or why should we discuss the use of thresholds?}\\[-0.1cm]{\footnotesize BES Macro (British Ecological Society Macroecology Special Interest Group Meeting) 2018, Saint-Andrews, Scotland}}\\[0.4cm]
févr 2017 & \parbox[t]{0.85\textwidth}{%
\textbf{Functional rarity of coral reef fishes across space (Best Presentation Award)}\\[-0.1cm]{\footnotesize Young Natural History Scientists’ Meeting, Paris, France}}\\[0.4cm]
sept 2016 & \parbox[t]{0.85\textwidth}{%
\textbf{A case study of Functional Rarity: worldwide coral reef fishes}\\[-0.1cm]{\footnotesize EcoSummit 2016, Montpellier, France}}\\[0.4cm]
\end{longtable}

\subsection{Articles de blog}\label{articles-de-blog}

\phantomsection\label{refs-e8e5b0644796f5cf65fc2be78a8ed8d4}
\begin{CSLReferences}{1}{1}
\bibitem[\citeproctext]{ref-Grenie_How_2022}
1. \textbf{Grenié, M.} (2022, December 6). How to Save Ggplot2 Plots in
a targets Workflow? Retrieved February 3, 2023, from
\url{https://ropensci.org/blog/2022/12/06/save-ggplot2-targets/}

\bibitem[\citeproctext]{ref-Salmon_Why_2022}
2. Salmon, M., \textbf{Grenié, M.}, \& Gruson, H. (2022, June 16). Why
You Should (or Shouldn't) Build an API Client. Retrieved February 16,
2023, from
\url{https://ropensci.org/blog/2022/06/16/publicize-api-client-yes-no/}

\bibitem[\citeproctext]{ref-Grenie_Best_2022}
3. \textbf{Grenié, M.}, Berti, E., Carvajal-Quintero, J., Dädlow, G. M.
L., Sagouis, A., \& Winter, M. (2022, March 2). Best practices for
taxonomic harmonization, an overlooked yet crucial step in biodiversity
analyses. Retrieved May 27, 2022, from
\url{https://methodsblog.com/2022/03/02/best-practices-for-taxonomic-harmonization-an-overlooked-yet-crucial-step-in-biodiversity-analyses/}

\bibitem[\citeproctext]{ref-Grenie_Community_2020}
4. \textbf{Grenié, M.}, \& Gruson, H. (2020, July 15). Community
Captioning of rOpenSci Community Calls. Retrieved May 27, 2022, from
\url{https://ropensci.org/blog/2020/07/15/subtitles/}

\bibitem[\citeproctext]{ref-Grenie_Access_2019}
5. \textbf{Grenié, M.}, \& Gruson, H. (2019, June 4). Access Publisher
Copyright \& Self-Archiving Policies via the 'SHERPA/RoMEO' API.
Retrieved May 27, 2022, from
\url{https://ropensci.org/blog/2019/06/04/rromeo/}

\end{CSLReferences}

\section{Enseignement}\label{enseignement}

Je suis formateur certifié de l'association
\href{https://carpentries.org/}{\emph{The Carpentries}} formant les
scientifiques à la programmation.

Au CEFE et à iDiv j'ai co-fondé des groupes d'utilisateur\(\cdotp\)rices
de R pour encourager une communauté de pratiques de R. \newline J'y ai
organisé et présenté plusieurs ateliers d'une heure sur des sujets liés
à R.

\begin{longtable}{@{\extracolsep{\fill}}ll}
2023-2024 (S1) & \parbox[t]{0.85\textwidth}{%
\textbf{Mathématiques/Statistiques}\hfill{\footnotesize L1 Biotechnologies (120 étudiant·es)}\newline
  47h (CM+TD) UGA\par%
  \vspace{0.1cm}\begin{minipage}{0.7\textwidth}%
\begin{itemize}%
\item Responsable d'UE, gestion des vacataires (3). Rappels de statistiques descriptives, introduction aux statistiques inférentielles et à R.%
\end{itemize}%
\end{minipage}%
\vspace{\parsep}}\\
2023-2024 (S2) & \parbox[t]{0.85\textwidth}{%
\textbf{Mathématiques/Statistiques 2}\hfill{\footnotesize L1 Biotechnologies (120 étudiant·es)}\newline
  24h (CM+TD) UGA\par%
  \vspace{0.1cm}\begin{minipage}{0.7\textwidth}%
\begin{itemize}%
\item Responsable d'UE, gestion des vacataires (3). Test non-paramétriques, tests d'ajustements.%
\end{itemize}%
\end{minipage}%
\vspace{\parsep}}\\
2023-2024 (S2) & \parbox[t]{0.85\textwidth}{%
\textbf{Biomathématiques/Statistiques 2}\hfill{\footnotesize L3 Biotechnologies (70 étudiant·es)}\newline
  35h (CM+TD) UGA\par%
  \vspace{0.1cm}\begin{minipage}{0.7\textwidth}%
\begin{itemize}%
\item Créateur et Responsable d'UE, gestion des vacataires (1). UE d'analyse de données basé sur l'exemple%
\end{itemize}%
\end{minipage}%
\vspace{\parsep}}\\
2023-2024 (S2) & \parbox[t]{0.85\textwidth}{%
\textbf{Outils Méthodologiques pour l'Analyse de Données en Santé}\hfill{\footnotesize M1 Ingénierie de la Santé (20 étudiant·es)}\newline
  25h (CM+TD) UGA\par%
  \vspace{0.1cm}\begin{minipage}{0.7\textwidth}%
\begin{itemize}%
\item Co-responsable et co-créateur de l'UE. UE de statistiques avancées. Introduction aux statistiques bayésiennes et la visualisation de données en santé, projet de statistiques%
\end{itemize}%
\end{minipage}%
\vspace{\parsep}}\\
2023-2024 (S1) & \parbox[t]{0.85\textwidth}{%
\textbf{Contrôle Qualité}\hfill{\footnotesize 3ème année de Pharmacie (100 étudiant·es)}\newline
  25h (TD)\par%
  \vspace{0.1cm}\begin{minipage}{0.7\textwidth}%
\begin{itemize}%
\item Responsable réalisation du projet de statistiques. Analyse de données en pratique.%
\end{itemize}%
\end{minipage}%
\vspace{\parsep}}\\
2023 & \parbox[t]{0.85\textwidth}{%
\textbf{Macroecology and macroevolution under global changes}\hfill{\footnotesize Master 'Biodiversity, Ecology and Evolution' (15 étudiant·es)}\newline
  2h (CM) UL\par%
  \vspace{0.1cm}\begin{minipage}{0.7\textwidth}%
\begin{itemize}%
\item Intervenant de cours sur l'harmonisation taxonomique%
\end{itemize}%
\end{minipage}%
\vspace{\parsep}}\\
2022 & \parbox[t]{0.85\textwidth}{%
\textbf{\href{https://www.nfdi4biodiversity.org/en/winterschool/}{École d'hiver de NFDI4Biodiversity sur la gestion de données en écologie et évolution}}\hfill{\footnotesize Masters, Doctorant·es, Post-doctorant·es (15 étudiant·es)}\newline
  2h (CM)\par%
  \vspace{0.1cm}\begin{minipage}{0.7\textwidth}%
\begin{itemize}%
\item Chargé de cours invité sur l'harmonisation taxonomique%
\end{itemize}%
\end{minipage}%
\vspace{\parsep}}\\
2022 & \parbox[t]{0.85\textwidth}{%
\textbf{Formation diversité et rareté fonctionnelle}\hfill{\footnotesize Masters, Doctorant·es, Post-doctorant·e, Chercheur·euses (20 personnes)}\newline
  4h (CM+TD) CESAB\par%
  \vspace{0.1cm}\begin{minipage}{0.7\textwidth}%
\begin{itemize}%
\item Co-organisateur d'une demie-journée de formation sur la diversité et la rareté fonctionnelle avec le soutien du Centre d'Analyse et de Synthèse sur la Biodiversité (CESAB) à Montpellier. \href{https://frbcesab.github.io/workshop-free/}{support de cours}%
\end{itemize}%
\end{minipage}%
\vspace{\parsep}}\\
2022 & \parbox[t]{0.85\textwidth}{%
\textbf{Introduction à git et GitHub}\hfill{\footnotesize Masters, Doctorant·es, Post-doctorant·e, Chercheur·euses (20 personnes)}\newline
  10h (CM+TD) UL\par%
  \vspace{0.1cm}\begin{minipage}{0.7\textwidth}%
\begin{itemize}%
\item Co-organisateur d'une formation d'une journée d'introduction à git et GitHub pour la recherche. \href{https://emilio-berti.github.io/idiv-git-introduction}{support de cours}%
\end{itemize}%
\end{minipage}%
\vspace{\parsep}}\\
2021, 2022 & \parbox[t]{0.85\textwidth}{%
\textbf{Macroecology and macroevolution under global changes}\hfill{\footnotesize Master 'Biodiversity, Ecology and Evolution' (20 étudiant·es)}\newline
  38h (CM+TD+TP) UL\par%
  \vspace{0.1cm}\begin{minipage}{0.7\textwidth}%
\begin{itemize}%
\item Chargé de cours les facettes de la biodiversité (taxonomique, phylogénétique et fonctionnelle). Montage et encadrement de séances de travaux pratiques sur R pour calculer ces indices. Encadrement de travaux de groupes en rapport avec les séances de cours.%
\end{itemize}%
\end{minipage}%
\vspace{\parsep}}\\
2018 & \parbox[t]{0.85\textwidth}{%
\textbf{Écologie Fonctionnelle}\hfill{\footnotesize Licence 3 — Biologie-Écologie (60 étudiant·es)}\newline
  15h (TP) UL\par%
  \vspace{0.1cm}\begin{minipage}{0.7\textwidth}%
\begin{itemize}%
\item Chargé de travaux pratiques expérimentaux sur l'écologie fonctionnelle. Encadrement des projets de groupes associés%
\end{itemize}%
\end{minipage}%
\vspace{\parsep}}\\
2017, 2018 & \parbox[t]{0.85\textwidth}{%
\textbf{Diversité Fonctionnelle}\hfill{\footnotesize Master Biologie, Écologie, Évolution}\newline
  8h (TD) UM\par%
  \vspace{0.1cm}\begin{minipage}{0.7\textwidth}%
\begin{itemize}%
\item Intervenant en cours pour une séance pratique sur la biogéographie fonctionnelle avec R. \href{https://github.com/Rekyt/functional_biogeo_practical}{support de cours}%
\end{itemize}%
\end{minipage}%
\vspace{\parsep}}\\
2016, 2017 & \parbox[t]{0.85\textwidth}{%
\textbf{Statistiques Descriptives}\hfill{\footnotesize Licence 2 – Sciences de la vie}\newline
  50h (TD) UM\par%
  \vspace{0.1cm}\begin{minipage}{0.7\textwidth}%
\begin{itemize}%
\item Chargé de TD de statistiques descriptives sur R%
\end{itemize}%
\end{minipage}%
\vspace{\parsep}}\\
\end{longtable}

\textbf{UGA} : Université Grenoble Alpes ; \textbf{UL} : Université de
Leipzig ; \textbf{UM}: Université de Montpellier

\section{Compétences}\label{compuxe9tences}

\subsection{Langues}\label{langues}

français {[}natif{]}; anglais {[}C2, courant et scientifique{]};
espagnol {[}B1-B2, quotidien{]}; allemand {[}A2-B1{]}; chinois
{[}A2-B1{]}; japonais {[}A2{]}

\subsection{Languages de Programmation et
Informatique}\label{languages-de-programmation-et-informatique}

\begin{itemize}
\tightlist
\item
  \faRProject~\textbf{R} ~\textbullet~ Meilleures pratiques de
  développement logiciel (tests unitaires avec\texttt{testthat}, site
  web \texttt{pkgdown}, intégration continue/déploiement continu,
  soumission au CRAN) ~\textbullet~ visualisation de données
  (\texttt{ggplot2} avancé et extensions) ~\textbullet~ manipulation de
  données spatiales {[}rasters \texttt{terra}, vecteurs \texttt{sf},
  base de données PostGIS, et interactions{]}
\item
  \faPython~\textbf{Python} ~\textbullet~ Programmation Orientée Objet
  ~\textbullet~ Tri de données avec \texttt{pandas}
\item
  \faDatabase~\textbf{SQL} interaction avec des bases de données
  ~\textbullet~ requêtes haute performance ~\textbullet~ interaction
  avec \texttt{R} ~\textbullet~ base de données spatiale
  PostgreSQL/PostGIS
\item
  \faServer~\textbf{Calcul de Haute Performance} soumission de
  \emph{job} sur SLURM/SGE ~\textbullet~ requêtes complexes
\item
  \faGit~contrôle de version ~\textbullet~ méthode git flow
  ~\textbullet~ développement collaboratif grâce à \faGithub~GitHub
\item
  Connaissances de base en \textbf{Julia}
\end{itemize}

\subsection{Statistiques et
modélisation}\label{statistiques-et-moduxe9lisation}

\begin{itemize}
\tightlist
\item
  Modèles nuls et de permutations pour la biodiversité
\item
  Modèles mixtes généralisés
\item
  \emph{Machine Learning}/\emph{Artificial Intelligence}: \emph{Random
  Forest}, \emph{Support Vector Machine}
\item
  Modélisation de population structurées en stade, modèle de
  Lotka-Volterra \emph{via} Mathematica
\end{itemize}

\end{document}
