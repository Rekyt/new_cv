%% Template for a CV
%% Author: Rob J Hyndman

\documentclass[10pt,a4paper,]{article}
\usepackage[scaled=0.86]{DejaVuSansMono}
\usepackage[sfdefault,lf,t]{carlito}

% Change color to blue
\usepackage{color,xcolor}
\definecolor{headcolor}{HTML}{990000}

\usepackage{ifxetex,ifluatex}
\usepackage{fixltx2e} % provides \textsubscript
\ifnum 0\ifxetex 1\fi\ifluatex 1\fi=0 % if pdftex
  \usepackage[T1]{fontenc}
  \usepackage[utf8]{inputenc}
\else % if luatex or xelatex
  \ifxetex
    \usepackage{mathspec}
  \else
    \usepackage{fontspec}
  \fi
  \defaultfontfeatures{Ligatures=TeX,Scale=MatchLowercase}
\fi

\usepackage[utf8]{inputenc}
\usepackage[T1]{fontenc}

% use upquote if available, for straight quotes in verbatim environments
\IfFileExists{upquote.sty}{\usepackage{upquote}}{}
% use microtype if available
\IfFileExists{microtype.sty}{%
\usepackage[]{microtype}
\UseMicrotypeSet[protrusion]{basicmath} % disable protrusion for tt fonts
}{}
\PassOptionsToPackage{hyphens}{url} % url is loaded by hyperref
\usepackage[unicode=true,hidelinks]{hyperref}
\urlstyle{same}  % don't use monospace font for urls
\usepackage{geometry}
\geometry{left=1.75cm,right=1.75cm,top=2.2cm,bottom=2cm}

\usepackage{longtable,booktabs}
% Fix footnotes in tables (requires footnote package)
\IfFileExists{footnote.sty}{\usepackage{footnote}\makesavenoteenv{long table}}{}
\IfFileExists{parskip.sty}{%
\usepackage{parskip}
}{% else
\setlength{\parindent}{0pt}
\setlength{\parskip}{6pt plus 2pt minus 1pt}
}
\setlength{\emergencystretch}{3em}  % prevent overfull lines
\providecommand{\tightlist}{%
  \setlength{\itemsep}{0pt}\setlength{\parskip}{0pt}}
\setcounter{secnumdepth}{0}

% set default figure placement to htbp
\makeatletter
\def\fps@figure{htbp}
\makeatother



\date{mars 2023}

\definecolor{headcolor}{HTML}{1E49BF}
\definecolor{linkscolor}{HTML}{026dbb}
\hypersetup{colorlinks=true, linkcolor=linkscolor, filecolor=linkscolor, urlcolor=linkscolor, urlbordercolor=linkscolor}

\usepackage{paralist,ragged2e,datetime}
\usepackage[hyphens]{url}
\usepackage{fancyhdr,enumitem,pifont}
\usepackage[compact,small,sf,bf]{titlesec}

\RaggedRight
\sloppy

% Header and footer
\pagestyle{fancy}
\makeatletter
\lhead{\sf\textcolor[gray]{0.4}{Curriculum Vitae: \@name}}
\rhead{\sf\textcolor[gray]{0.4}{\thepage}}
\cfoot{}
\def\headrule{{\color[gray]{0.4}\hrule\@height\headrulewidth\@width\headwidth \vskip-\headrulewidth}}
\makeatother

% Header box
\usepackage{tabularx}

\makeatletter
\def\name#1{\def\@name{#1}}
\def\info#1{\def\@info{#1}}
\makeatother
\newcommand{\shadebox}[3][.9]{\fcolorbox[gray]{0}{#1}{\parbox{#2}{#3}}}

\usepackage{calc}
\newlength{\headerboxwidth}
\setlength{\headerboxwidth}{\textwidth}
%\addtolength{\headerboxwidth}{0.2cm}
\makeatletter
\def\maketitle{
\thispagestyle{plain}
\vspace*{-1.4cm}
\shadebox[0.9]{\headerboxwidth}{\sf\color{headcolor}\hfil
\hbox to 0.98\textwidth{\begin{tabular}{l}
\\[-0.3cm]
\LARGE\textbf{\@name}
\\[0.1cm]\large Chercheur Postdoctoral\\[0.6cm]
\normalsize\textbf{Curriculum Vitae}\\
\normalsize mars 2023
\end{tabular}
\hfill\hbox{\fontsize{9}{12}\sf
\begin{tabular}{@{}rl@{}}
\@info
\end{tabular}}}\hfil
}
\vspace*{0.2cm}}
\makeatother

% Section headings
\titlelabel{}
\titlespacing{\section}{0pt}{1.5ex}{0.5ex}
\titleformat*{\section}{\color{headcolor}\large\sf\bfseries}
\titleformat*{\subsection}{\color{headcolor}\sf\bfseries}
\titlespacing{\subsection}{0pt}{1ex}{0.5ex}

% Miscellaneous dimensions
\setlength{\parskip}{0ex}
\setlength{\parindent}{0em}
\setlength{\headheight}{15pt}
\setlength{\tabcolsep}{0.15cm}
\clubpenalty = 10000
\widowpenalty = 10000
\setlist{itemsep=1pt}
\setdescription{labelwidth=1.2cm,leftmargin=1.5cm,labelindent=1.5cm,font=\rm}

% Make nicer bullets
\renewcommand{\labelitemi}{\ding{228}}

\usepackage{booktabs,fontawesome5}
%\usepackage[t1,scale=0.86]{sourcecodepro}

\name{Matthias Grenié}
\def\imagetop#1{\vtop{\null\hbox{#1}}}
\info{%
\raisebox{-0.05cm}{\imagetop{\faIcon{map-marker-alt}}} &  \imagetop{\begin{tabular}{@{}l@{}}German
Centre for Integrative Biodversity Research (iDiv),
Halle-Jena-Leipzig\end{tabular}}\\ %
\faIcon{home} & \href{http://rekyt.github.io}{rekyt.github.io}\\% %
\faIcon{phone-alt} & +49 341 9733176\\%
\faIcon{envelope} & \href{mailto:matthias.grenie@idiv.de}{\nolinkurl{matthias.grenie@idiv.de}}\\%
\faIcon{twitter} & \href{https://twitter.com/LeNematode}{@LeNematode}\\%
\faIcon{github} & \href{https://github.com/Rekyt}{Rekyt}\\%
%
\faIcon{google} & \href{https://scholar.google.com/citations?user=fZ1\_d7QAAAAJ}{fZ1\_d7QAAAAJ}\\%
\faIcon{orcid} & \href{https://orcid.org/0000-0002-4659-7522}{0000-0002-4659-7522}\\%
}


%\usepackage{inconsolata}


\setlength\LTleft{0pt}
\setlength\LTright{0pt}

% Pandoc CSL macros
\newlength{\cslhangindent}
\setlength{\cslhangindent}{1.5em}
\newlength{\csllabelwidth}
\setlength{\csllabelwidth}{3em}
\newenvironment{CSLReferences}[3] % #1 hanging-ident, #2 entry spacing
 {% don't indent paragraphs
  \setlength{\parindent}{0pt}
  % turn on hanging indent if param 1 is 1
  \ifodd #1 \everypar{\setlength{\hangindent}{\cslhangindent}}\ignorespaces\fi
  % set entry spacing
  \ifnum #2 > 0
  \setlength{\parskip}{#2\baselineskip}
  \fi
 }%
 {}
\usepackage{calc}
\newcommand{\CSLBlock}[1]{#1\hfill\break}
\newcommand{\CSLLeftMargin}[1]{\parbox[t]{\csllabelwidth}{\hfill #1~}}
\newcommand{\CSLRightInline}[1]{\parbox[t]{\linewidth - \cslhangindent - \csllabelwidth}{#1}\vspace{0.8ex}}
\newcommand{\CSLIndent}[1]{\hspace{\cslhangindent}#1}


\def\endfirstpage{\newpage}

\begin{document}
\maketitle


\hypertarget{cursus-universitaire}{%
\section{Cursus Universitaire}\label{cursus-universitaire}}

\begin{longtable}{@{\extracolsep{\fill}}ll}
2016-2020 & \parbox[t]{0.85\textwidth}{%
\textbf{Doctorat en Écologie}\hfill{\footnotesize École Doctorale 584 GAIA, Univ. Montpellier}\newline
  Montpellier, France\par%
  \vspace{0.1cm}\begin{minipage}{0.7\textwidth}%
\begin{itemize}%
\item Directeurs: \textbf{Prof. Françoiz Munoz} et \textbf{Dr. Cyrille Violle}\\~Titre: « En dehors de la norme: déviation de l'optimalité écologique et rareté fonctionnelle »%
\end{itemize}%
\end{minipage}%
\vspace{\parsep}}\\
2013-2015 & \parbox[t]{0.85\textwidth}{%
\textbf{Master Biosciences}\hfill{\footnotesize École Normale Supérieure de Lyon, Univ. Claude Bernard Lyon 1}\newline
  Lyon, France\par%
  \vspace{0.1cm}\begin{minipage}{0.7\textwidth}%
\begin{itemize}%
\item Normalien élève%
\end{itemize}%
\end{minipage}%
\vspace{\parsep}}\\
2012-2013 & \parbox[t]{0.85\textwidth}{%
\textbf{Licence Biosciences}\hfill{\footnotesize École Normale Supérieure de Lyon, Univ. Claude Bernard Lyon 1}\newline
  Lyon, France\par%
  \vspace{0.1cm}\begin{minipage}{0.7\textwidth}%
\begin{itemize}%
\item Normalien élève%
\end{itemize}%
\end{minipage}%
\vspace{\parsep}}\\
2010-2012 & \parbox[t]{0.85\textwidth}{%
\textbf{Classe Préparatoire BCPST}\hfill{\footnotesize Lycée Saint-Louis}\newline
  Paris, France\par%
  \vspace{0.1cm}\begin{minipage}{0.7\textwidth}%
\begin{itemize}%
\item Admis au concours Agronomie et ENS Lyon%
\end{itemize}%
\end{minipage}%
\vspace{\parsep}}\\
\end{longtable}

\hypertarget{parcours-professionnel-acaduxe9mique}{%
\section{Parcours Professionnel
Académique}\label{parcours-professionnel-acaduxe9mique}}

\begin{longtable}{@{\extracolsep{\fill}}ll}
oct. 2020 - \textbf{présent} & \parbox[t]{0.85\textwidth}{%
\textbf{Chercheur Postdoctoral}\hfill{\footnotesize German Center for Integrative Biodiversity Research (iDiv) / Leipzig University}\newline
  Leipzig, Allemagne\par%
  \vspace{0.1cm}\begin{minipage}{0.7\textwidth}%
\begin{itemize}%
\item Supervisé par \textbf{Dr. Marten Winter} sur la biogéographie fonctionnelle des plantes introduites%
\end{itemize}%
\end{minipage}%
\vspace{\parsep}}\\
2016-2020 & \parbox[t]{0.85\textwidth}{%
\textbf{Doctorant}\hfill{\footnotesize Université de Montpellier}\newline
  Montpellier, France\par%
  \vspace{0.1cm}\begin{minipage}{0.7\textwidth}%
\begin{itemize}%
\item Directeurs: François Munoz et Cyrille Violle. Contrat doctoral spécifique normalien. Sur la rareté fonctionnelle et l'écologie fonctionnelle des communautés%
\end{itemize}%
\end{minipage}%
\vspace{\parsep}}\\
févr. 2015 - juin 2015 & \parbox[t]{0.85\textwidth}{%
\textbf{Stage de Master 2}\hfill{\footnotesize EcoFOG}\newline
  Kourou, Guyane Française, France\par%
  \vspace{0.1cm}\begin{minipage}{0.7\textwidth}%
\begin{itemize}%
\item Encadrant: \textbf{Dr. Bruno Hérault} sur la variabilité intraspécifique de croissance des arbres tropicaux en fonction de leurs traits fonctionnels%
\end{itemize}%
\end{minipage}%
\vspace{\parsep}}\\
sept. 2014 - déc. 2014 & \parbox[t]{0.85\textwidth}{%
\textbf{Stage de Master 2}\hfill{\footnotesize ISEM, Université de Montpellier}\newline
  Montpellier, France\par%
  \vspace{0.1cm}\begin{minipage}{0.7\textwidth}%
\begin{itemize}%
\item Encadrante: \textbf{Dr. Ophélie Ronce} sur la prédiction de l'adaptation au changement climatique des arbres en utilisant des modèles de génétique quantitative de populations structurées en stade%
\end{itemize}%
\end{minipage}%
\vspace{\parsep}}\\
févr. 2014 - juill. 2014 & \parbox[t]{0.85\textwidth}{%
\textbf{Stage de Master 2}\hfill{\footnotesize Indiana University}\newline
  Bloomington, IN, États-Unis\par%
  \vspace{0.1cm}\begin{minipage}{0.7\textwidth}%
\begin{itemize}%
\item Encadrent: \textbf{Dr. Jean-François Goût} dans l'équipe du \textbf{Pr. Michael Lynch} sur les motifs génomiques au sein du complexe \textit{Paramecium}%
\end{itemize}%
\end{minipage}%
\vspace{\parsep}}\\
juin 2013 - août 2013 & \parbox[t]{0.85\textwidth}{%
\textbf{Stage de L3}\hfill{\footnotesize Université Pierre et Marie Curie Paris 6}\newline
  Paris, France\par%
  \vspace{0.1cm}\begin{minipage}{0.7\textwidth}%
\begin{itemize}%
\item Encadrants: \textbf{Dr. Éric Bapteste} et \textbf{Prof. Philippe Lopez} sur réseaux de similarités de séquences en fonction de l'origine évolutive des voies métaboliques chez \textit{Chlamydomonas reinhardtii}%
\end{itemize}%
\end{minipage}%
\vspace{\parsep}}\\
\end{longtable}

\newpage

\hypertarget{encadrement-scientifique}{%
\section{Encadrement Scientifique}\label{encadrement-scientifique}}

\begin{longtable}{@{\extracolsep{\fill}}ll}
2022-\textbf{présent} & \parbox[t]{0.85\textwidth}{%
\textbf{Membre du comité de thèse de \href{https://www.idiv.de/en/profile/1248.html}{Rachel Souza Fereira}}\\[-0.1cm]{\footnotesize }}\\[0.4cm]
avr. 2018 - mai 2018 & \parbox[t]{0.85\textwidth}{%
\textbf{Encadrement stage de M1}\\[-0.1cm]{\footnotesize Charlotte Guérineau, stage sur la rareté fonctionnelle de différents groupes taxonomiques (oiseaux, plantes, poissons, et bactéries du sol) en France}}\\[0.4cm]
avr. 2019 - juin 2019 & \parbox[t]{0.85\textwidth}{%
\textbf{Co-encadrement stage de M1}\\[-0.1cm]{\footnotesize Nathan Mazet, stage sur les stratégies alimentaires des oiseaux à l'échelle globale, encadrant principal: Pr. Jean-Yves Barnagaud}}\\[0.4cm]
\end{longtable}

\hypertarget{activituxe9s-dorganisation-et-responsabilituxe9s-collectives}{%
\section{Activités d'organisation et responsabilités
collectives}\label{activituxe9s-dorganisation-et-responsabilituxe9s-collectives}}

\begin{longtable}{@{\extracolsep{\fill}}ll}
2022-\textbf{présent} & \parbox[t]{0.85\textwidth}{%
\textbf{Relecteur de logiciel scientifique pour rOpenSci}\\[-0.1cm]{\footnotesize Pour le paquet \href{https://github.com/ropensci/software-review/issues/505}{\texttt{npi}}}}\\[0.4cm]
2022-\textbf{présent} & \parbox[t]{0.85\textwidth}{%
\textbf{Co-fondateur et animateur de \textit{iCode}}\\[-0.1cm]{\footnotesize Groupe d'animation et d'entraide (forum d'échanges, séminaires) sur R et la programmation à iDiv}}\\[0.4cm]
2022-\textbf{présent} & \parbox[t]{0.85\textwidth}{%
\textbf{Représentant des post-doctorant$\cdotp$es (130 personnes)}\\[-0.1cm]{\footnotesize Au \href{https://www.idiv.de/en/council.html}{\textbf{conseil de laboratoire d'iDiv}}, équivalent du conseil d'UMR}}\\[0.4cm]
2021-\textbf{présent} & \parbox[t]{0.85\textwidth}{%
\textbf{Organisateur des séminaires de l'équipe sDiv}\\[-0.1cm]{\footnotesize toutes les deux semaines}}\\[0.4cm]
2018-2020 & \parbox[t]{0.85\textwidth}{%
\textbf{Représentant des Doctorant$\cdotp$es (70 personnes)}\\[-0.1cm]{\footnotesize Conseil d'UMR du CEFE}}\\[0.4cm]
2018 & \parbox[t]{0.85\textwidth}{%
\textbf{Co-organisateur d'un atelier sur les pratiques d'\textit{Open Access}}\\[-0.1cm]{\footnotesize Un événement d'une demi-journée pour apprendre à rendre nos articles accessibles, pour \textasciitilde{}50 personnes au CEFE à Montpellier}}\\[0.4cm]
2017-\textbf{présent} & \parbox[t]{0.85\textwidth}{%
\textbf{Relecteur de journaux scientifiques (\href{https://publons.com/wos-op/researcher/1466931/matthias-grenie}{voir le profil Publons})}\\[-0.1cm]{\footnotesize Pour \textit{Ecology Letters}, \textit{Functional Ecology}, \textit{Biological Reviews}, \textit{PeerJ}, \textit{Biological Conservation}, \textit{New Phytologist}, \textit{Journal of Applied Ecology}, et \textit{Journal of Biogeography}}}\\[0.4cm]
2017 & \parbox[t]{0.85\textwidth}{%
\textbf{Co-organisateur de la conférence Jeune Chercheur$\cdotp$euses \href{https://web.archive.org/web/20180823012054/http://www.mee.univ-montp2.fr/editions-precedentes/edition-2017/organisateurs-2017/}{'Models in Ecology and Evolution'}}\\[-0.1cm]{\footnotesize Une conférence annuelle pour jeunes chercheur$\cdotp$euses avec un public de \textasciitilde{}100 personnes du Master au Post-doctorat}}\\[0.4cm]
2016-2020 & \parbox[t]{0.85\textwidth}{%
\textbf{Fondateur et animateur du groupe d'utilisateur$\cdotp$rices de R au CEFE}\\[-0.1cm]{\footnotesize Animation mensuelle pour les personnes utilisant R au CEFE, incluant présentations et ateliers pratiques}}\\[0.4cm]
\end{longtable}

\hypertarget{participation-uxe0-des-projets-scientifiques}{%
\section{Participation à des projets
scientifiques}\label{participation-uxe0-des-projets-scientifiques}}

\begin{longtable}{@{\extracolsep{\fill}}ll}
2022-\textbf{présent} & \parbox[t]{0.85\textwidth}{%
\textbf{Groupe \href{https://www.fondationbiodiversite.fr/la-frb-en-action/programmes-et-projets/le-cesab/impacts/}{IMPACTS}}\hfill{\footnotesize Centre de Synthèse et d'Analyse sur la Biodiversité (CESAB)}\newline
  \empty%
  \vspace{0.1cm}\begin{minipage}{0.7\textwidth}%
\begin{itemize}%
\item Groupe de synthèse sur le suivi temporel de la biodiversité terrestre en France\break Membre du groupe, en charge de la gestion de données\break Première réunion en présentiel : mars 2023%
\end{itemize}%
\end{minipage}%
\vspace{\parsep}}\\
2021-\textbf{présent} & \parbox[t]{0.85\textwidth}{%
\textbf{Groupe \href{https://glonaf.org/}{Global Naturalized Alien Flora}}\hfill{\footnotesize GloNAF}\newline
  \empty%
  \vspace{0.1cm}\begin{minipage}{0.7\textwidth}%
\begin{itemize}%
\item Groupe européen sur les invasions végétales \textit{via} la base GloNAF \break Contributeur aux réunions bisannuelles du groupe \break Premier auteur d'un manuscrit pour le groupe -- Co-auteur d'un manuscrit%
\end{itemize}%
\end{minipage}%
\vspace{\parsep}}\\
2017-\textbf{présent} & \parbox[t]{0.85\textwidth}{%
\textbf{\href{https://www.fondationbiodiversite.fr/la-frb-en-action/programmes-et-projets/le-cesab/free/}{Groupe FREE}}\hfill{\footnotesize CESAB}\newline
  \empty%
  \vspace{0.1cm}\begin{minipage}{0.7\textwidth}%
\begin{itemize}%
\item Groupe de synthèse sur la rareté fonctionnelle en écologie and évolution \break Membre co-fondateur du groupe \break Premier auteur de 3 articles liés au groupe -- Co-auteur de 4 articles%
\end{itemize}%
\end{minipage}%
\vspace{\parsep}}\\
\end{longtable}

\hypertarget{bourses-et-financements}{%
\section{Bourses et Financements}\label{bourses-et-financements}}

\begin{longtable}{@{\extracolsep{\fill}}ll}
2023-\textbf{présent} & \parbox[t]{0.85\textwidth}{%
\textbf{Financement}\hfill{\footnotesize Flexpool (interne iDiv) 10k€}\newline
  \empty%
  \vspace{0.1cm}\begin{minipage}{0.7\textwidth}%
\begin{itemize}%
\item Co-rédacteur et collaborateur du projet mené par Dr. Qiang Yang sur l'impact des espèces de plantes naturalisées sur les réseaux plantes-pollinisateurs le long de gradients altitudinaux%
\end{itemize}%
\end{minipage}%
\vspace{\parsep}}\\
2022-\textbf{présent} & \parbox[t]{0.85\textwidth}{%
\textbf{Financement}\hfill{\footnotesize Flexpool (interne iDiv) 10k€}\newline
  \empty%
  \vspace{0.1cm}\begin{minipage}{0.7\textwidth}%
\begin{itemize}%
\item Co-rédacteur et collaborateur du projet mené par Dr. Bettina Ohse, sur le lien entre traits fonctionnels et taux démographiques des arbres%
\end{itemize}%
\end{minipage}%
\vspace{\parsep}}\\
Sept. 2016 & \parbox[t]{0.85\textwidth}{%
\textbf{Bourse doctorale}\hfill{\footnotesize École Normale Supérieure de Lyon}\newline
  \empty%
  \vspace{0.1cm}\begin{minipage}{0.7\textwidth}%
\begin{itemize}%
\item Contrat Doctoral Spécifique Normalien%
\end{itemize}%
\end{minipage}%
\vspace{\parsep}}\\
\end{longtable}

\hypertarget{contributions-scientifiques}{%
\section{Contributions
scientifiques}\label{contributions-scientifiques}}

\faFile*~14 publications (5 en premier auteur). \faQuoteLeft~457
citations (Google Scholar). \faHSquare~h-index 10 (Google Scholar).
\faRProject~6 paquets R

\hypertarget{publications}{%
\subsection{Publications}\label{publications}}

\hypertarget{acceptuxe9e}{%
\subsubsection{Acceptée}\label{acceptuxe9e}}

\hypertarget{bibliography}{}
\leavevmode\vadjust pre{\hypertarget{ref-Munoz_ecological_2023}{}}%
1. Munoz, F., Klausmeier, C., \ldots, \textbf{Grenié, M.}, Loiseau, N.,
Mahaut, L., Maire, A., Mouillot, D., Violle, C., \& Kraft, N. (2023).
The ecological causes of functional distinctiveness in communities.
\emph{Accepted in Ecology Letters}.
\url{https://doi.org/10.22541/au.166488862.28762630/v1}

\hypertarget{publiuxe9es}{%
\subsubsection{Publiées}\label{publiuxe9es}}

\hypertarget{bibliography}{}
\leavevmode\vadjust pre{\hypertarget{ref-Cutts_Links_2023}{}}%
1. Cutts, V., Hanz, D. M., Barajas-Barbosa, M. P., Schrodt, F.,
Steinbauer, M. J., Beierkuhnlein, C., Denelle, P., Fernández-Palacios,
J. M., Gaüzère, P., \textbf{Grenié, M.}, Irl, S. D. H., \ldots{} Algar,
A. C. (2023). Links to rare climates do not translate into distinct
traits for island endemics. \emph{Ecology Letters}, \emph{n/a}(n/a).
\url{https://doi.org/10.1111/ele.14169}

\leavevmode\vadjust pre{\hypertarget{ref-Grenie_fundiversity_2023}{}}%
2. \textbf{Grenié, M.}, \& Gruson, H. (2023). Fundiversity: A modular R
package to compute functional diversity indices. \emph{Ecography},
\emph{n/a}(n/a), e06585. \url{https://doi.org/10.1111/ecog.06585}

\leavevmode\vadjust pre{\hypertarget{ref-Gauzere_functional_2022}{}}%
3. Gaüzère, P., Denelle, P., Fournier, B., \textbf{Grenié, M.},
Delalandre, L., Münkemüller, T., Munoz, F., Violle, C., \& Thuiller, W.
(2022). The functional trait distinctiveness of plant species is scale
dependent. \emph{Ecography}, \emph{in press}.

\leavevmode\vadjust pre{\hypertarget{ref-Grenie_Harmonizing_2022}{}}%
4. \textbf{Grenié, M.}, Berti, E., Carvajal-Quintero, J., Dädlow, G. M.
L., Sagouis, A., \& Winter, M. (2022). Harmonizing taxon names in
biodiversity data: A review of tools, databases and best practices.
\emph{Methods in Ecology and Evolution}, 2041--210X.13802.
\url{https://doi.org/10.1111/2041-210X.13802}

\leavevmode\vadjust pre{\hypertarget{ref-Jurburg_community_2022}{}}%
5. Jurburg, S. D., Buscot, F., Chatzinotas, A., Chaudhari, N. M., Clark,
A. T., Garbowski, M., \textbf{Grenié, M.}, Hom, E. F. Y., Karakoç, C.,
Marr, S., Neumann, S., \ldots{} Heintz-Buschart, A. (2022). The
community ecology perspective of omics data. \emph{Microbiome},
\emph{10}(1), 225. \url{https://doi.org/10.1186/s40168-022-01423-8}

\leavevmode\vadjust pre{\hypertarget{ref-Mouillot_dimensionality_2021}{}}%
6. Mouillot, D., Loiseau, N., \textbf{Grenié, M.}, Algar, A. C.,
Allegra, M., Cadotte, M. W., Casajus, N., Denelle, P., Guéguen, M.,
Maire, A., Maitner, B., \ldots{} Auber, A. (2021). The dimensionality
and structure of species trait spaces. \emph{Ecology Letters},
\emph{24}(9), 1988--2009. \url{https://doi.org/10.1111/ele.13778}

\leavevmode\vadjust pre{\hypertarget{ref-Grenie_prediction_2020}{}}%
7. \textbf{Grenié, M.}, Violle, C., \& Munoz, F. (2020). Is prediction
of species richness from stacked species distribution models biased by
habitat saturation? \emph{Ecological Indicators}, \emph{111}, 105970.
\url{https://doi.org/10.1016/j.ecolind.2019.105970}

\leavevmode\vadjust pre{\hypertarget{ref-Loiseau_Global_2020}{}}%
8. Loiseau, N., Mouquet, N., Casajus, N., \textbf{Grenié, M.}, Guéguen,
M., Maitner, B., Mouillot, D., Ostling, A., Renaud, J., Tucker, C.,
Velez, L., \ldots{} Violle, C. (2020). Global distribution and
conservation status of ecologically rare mammal and bird species.
\emph{Nature Communications}, \emph{11}(1, 1), 5071.
\url{https://doi.org/10.1038/s41467-020-18779-w}

\leavevmode\vadjust pre{\hypertarget{ref-Barnagaud_Functional_2019}{}}%
9. Barnagaud, J.-Y., Mazet, N., Munoz, F., \textbf{Grenié, M.}, Denelle,
P., Sobral, M., Kissling, W. D., Sekercioglu, Ç. H., \& Violle, C.
(2019). Functional biogeography of dietary strategies in birds.
\emph{Global Ecology and Biogeography}, \emph{28}(7), 1004--1017.
\url{https://doi.org/10.1111/geb.12910}

\leavevmode\vadjust pre{\hypertarget{ref-Joffard_Effect_2019}{}}%
10. Joffard, N., Massol, F., \textbf{Grenié, M.}, Montgelard, C., \&
Schatz, B. (2019). Effect of pollination strategy, phylogeny and
distribution on pollination niches of Euro-Mediterranean orchids.
\emph{Journal of Ecology}, \emph{107}(1), 478--490.
\url{https://doi.org/10.1111/1365-2745.13013}

\leavevmode\vadjust pre{\hypertarget{ref-Grenie_Functional_2018}{}}%
11. \textbf{Grenié, M.}, Mouillot, D., Villéger, S., Denelle, P.,
Tucker, C. M., Munoz, F., \& Violle, C. (2018). Functional rarity of
coral reef fishes at the global scale: Hotspots and challenges for
conservation. \emph{Biological Conservation}, \emph{226}, 288--299.
\url{https://doi.org/10.1016/j.biocon.2018.08.011}

\leavevmode\vadjust pre{\hypertarget{ref-Munoz_ecolottery_2018}{}}%
12. Munoz, F., \textbf{Grenié, M.}, Denelle, P., Taudière, A., Laroche,
F., Tucker, C., \& Violle, C. (2018). Ecolottery: Simulating and
assessing community assembly with environmental filtering and neutral
dynamics in R. \emph{Methods in Ecology and Evolution}, \emph{9}(3),
693--703. \url{https://doi.org/10.1111/2041-210X.12918}

\leavevmode\vadjust pre{\hypertarget{ref-Grenie_funrar_2017}{}}%
13. \textbf{Grenié, M.}, Denelle, P., Tucker, C. M., Munoz, F., \&
Violle, C. (2017). Funrar: An R package to characterize functional
rarity. \emph{Diversity and Distributions}, \emph{23}(12), 1365--1371.
\url{https://doi.org/10.1111/ddi.12629}

\leavevmode\vadjust pre{\hypertarget{ref-Violle_Common_2017}{}}%
14. Violle, C., Thuiller, W., Mouquet, N., Munoz, F., Kraft, N. J. B.,
Cadotte, M. W., Livingstone, S. W., \textbf{Grenié, M.}, \& Mouillot, D.
(2017). A Common Toolbox to Understand, Monitor or Manage Rarity? A
Response to Carmona et al. \emph{Trends in Ecology \& Evolution},
\emph{32}(12), 891--893.
\url{https://doi.org/10.1016/j.tree.2017.10.001}

\hypertarget{paquets-r}{%
\subsection{Paquets R}\label{paquets-r}}

\begin{itemize}
\item
  \textbf{\texttt{funrar}} (créateur principal et mainteneux)
  \hfill\break calcule des indices de rareté fonctionnelle, central pour
  les travaux du groupe CESAB FREE
  \hfill\break ~\href{https://doi.org/10.1111/ddi.12629}{\faFile*~
  article publié} --
  \href{https://cran.r-project.org/package=funrar}{\faRProject~CRAN} --
  \href{https://github.com/Rekyt/funrar}{\faGithub~GitHub}
\item
  \textbf{\texttt{ecolottery}} (co-créateur) \hfill\break simule de
  manière efficace des processus d'assemblages de communautés en
  intégrant à la fois des processus neutre et de niches
  \hfill\break ~\href{https://doi.org/10.1111/2041-210X.12918}{\faFile*~article
  publié} --
  \href{https://cran.r-project.org/package=ecolottery}{\faRProject~CRAN}
  -- \href{https://github.com/frmunoz/ecolottery}{\faGithub~GitHub}
\item
  \textbf{\texttt{fundiversity}} (co-créateur et mainteneur)
  \hfill\break calcul des indices de diversité fonctionnelle de manière
  efficace, modulaire, en intégrant des fonctionnalités modernes
  récentes (parallélisation, mémoisation)
  \hfill\break ~\href{https://doi.org/10.1111/ecog.06585}{\faFile*~article
  publié} --
  \href{https://cran.r-project.org/package=fundiversity}{\faRProject~CRAN}
  -- \href{https://github.com/bisaloo/fundiversity}{\faGithub~GitHub}
\item
  \textbf{\texttt{funbiogeo}} (co-créateur et mainteneur)
  \hfill\break facilite les analyses en biogéographique fonctionnelle,
  dévelopé dans le contexte du groupe CESAB FREE
  \hfill\break ~\href{https://github.com/FRBCesab/funbiogeo}{\faGithub~GitHub}
\item
  \textbf{\texttt{rtaxref}} (créateur et maintenuer) \hfill\break accède
  aux données de l'API du Référentiel Taxonomique Français (TAXREF)
  \hfill\break ~\href{https://github.com/Rekyt/rtaxref}{\faGithub~GitHub}
\item
  Contributions significatives à d'autres paquets R:
  \href{https://cran.r-project.org/package=taxize}{\textbf{\texttt{taxize}}}
  (extraction et standardisation de données taxonomiques),
  \href{https://cran.r-project.org/package=ggfortify}{\textbf{\texttt{ggfortify}}}
  (simplification et extraction automatique de graphiques), et
  \href{https://cran.r-project.org/package=traitdataform}{\textbf{\texttt{traitdataform}}}
  (aide à l'écriture de métadonnées sur les traits fonctionnels)
\end{itemize}

\hypertarget{actes-de-confuxe9rences}{%
\subsection{Actes de Conférences}\label{actes-de-confuxe9rences}}

\hypertarget{bibliography}{}
\leavevmode\vadjust pre{\hypertarget{ref-Grenie_Matching_2021}{}}%
1. \textbf{Grenié, M.}, Berti, E., Carvajal-Quintero, J., Winter, M., \&
Sagouis, A. (2021). Matching Species Names Across Biodiversity
Databases: Sources, tools, pitfalls and best practices for taxonomic
harmonization. \emph{Biodiversity Information Science and Standards},
\emph{5}, e75359. \url{https://doi.org/10.3897/biss.5.75359}

\hypertarget{pruxe9sentations-invituxe9es}{%
\subsection{Présentations Invitées}\label{pruxe9sentations-invituxe9es}}

\begin{longtable}{@{\extracolsep{\fill}}ll}
déc. 2021 & \parbox[t]{0.85\textwidth}{%
\textbf{Séminaire de l'équipe de \href{https://www.researchgate.net/lab/Holger-Krefts-lab-Holger-Kreft}{Holger Kreft} sur 'Navigating the landscape of taxonomic harmonization'}\\[-0.1cm]{\footnotesize Université de Göttingen, Allemagne}}\\[0.4cm]
mai 2022 & \parbox[t]{0.85\textwidth}{%
\textbf{Présentation invitée à la 3ème réunion de \href{https://d2kab.mystrikingly.com/}{Data to Knowledge in Agronomy and Biodiversity (D2KAB)} sur 'Taxonomic Databases of Plants and Animals'}\\[-0.1cm]{\footnotesize Paris/en visioconférence}}\\[0.4cm]
\end{longtable}

\hypertarget{pruxe9sentations-orales}{%
\subsection{Présentations Orales}\label{pruxe9sentations-orales}}

\begin{longtable}{@{\extracolsep{\fill}}ll}
sept 2022 & \parbox[t]{0.85\textwidth}{%
\textbf{A barrier to global plant invasion ecology: gaps in trait availability for alien species}\\[-0.1cm]{\footnotesize Neobiota 2022, Tartu, Estonia}}\\[0.4cm]
juil 2022 & \parbox[t]{0.85\textwidth}{%
\textbf{A barrier to global plant invasion ecology: gaps in trait availability for alien species}\\[-0.1cm]{\footnotesize BES Macro 2022, Online}}\\[0.4cm]
avr 2022 & \parbox[t]{0.85\textwidth}{%
\textbf{A barrier to global plant invasion ecology: gaps in trait availability for alien species}\\[-0.1cm]{\footnotesize iDiv Conference 2022, Leipzig, Germany}}\\[0.4cm]
août 2021 & \parbox[t]{0.85\textwidth}{%
\textbf{Navigating the landscape of taxonomic harmonization: data, tools, and best practices}\\[-0.1cm]{\footnotesize GfÖ conference (German Speaking Ecological Society Meeting) 2021}}\\[0.4cm]
juil 2021 & \parbox[t]{0.85\textwidth}{%
\textbf{Navigating the landscape of taxonomic harmonization: data, tools, and best practices}\\[-0.1cm]{\footnotesize BES Macro 2021, Online}}\\[0.4cm]
oct 2018 & \parbox[t]{0.85\textwidth}{%
\textbf{Functional rarity of coral reef fishes at the global scale: Hotspots and challenges for conservation}\\[-0.1cm]{\footnotesize French Ecological Society Meeting (SFÉcologie) 2018, Rennes, France}}\\[0.4cm]
juil 2018 & \parbox[t]{0.85\textwidth}{%
\textbf{Predicting Species Richness with unicorns or why should we discuss the use of thresholds?}\\[-0.1cm]{\footnotesize BES Macro (British Ecological Society Macroecology Special Interest Group Meeting) 2018, Saint-Andrews, Scotland}}\\[0.4cm]
févr 2017 & \parbox[t]{0.85\textwidth}{%
\textbf{Functional rarity of coral reef fishes across space (Best Presentation Award)}\\[-0.1cm]{\footnotesize Young Natural History Scientists’ Meeting, Paris, France}}\\[0.4cm]
sept 2016 & \parbox[t]{0.85\textwidth}{%
\textbf{A case study of Functional Rarity: worldwide coral reef fishes}\\[-0.1cm]{\footnotesize EcoSummit 2016, Montpellier, France}}\\[0.4cm]
\end{longtable}

\hypertarget{articles-de-blog}{%
\subsection{Articles de blog}\label{articles-de-blog}}

\hypertarget{bibliography}{}
\leavevmode\vadjust pre{\hypertarget{ref-Grenie_How_2022}{}}%
1. \textbf{Grenié, M.} (2022, December 6). How to Save Ggplot2 Plots in
a targets Workflow? Retrieved February 3, 2023, from
\url{https://ropensci.org/blog/2022/12/06/save-ggplot2-targets/}

\leavevmode\vadjust pre{\hypertarget{ref-Salmon_Why_2022}{}}%
2. Salmon, M., \textbf{Grenié, M.}, \& Gruson, H. (2022, June 16). Why
You Should (or Shouldn't) Build an API Client. Retrieved February 16,
2023, from
\url{https://ropensci.org/blog/2022/06/16/publicize-api-client-yes-no/}

\leavevmode\vadjust pre{\hypertarget{ref-Grenie_Best_2022}{}}%
3. \textbf{Grenié, M.}, Berti, E., Carvajal-Quintero, J., Dädlow, G. M.
L., Sagouis, A., \& Winter, M. (2022, March 2). Best practices for
taxonomic harmonization, an overlooked yet crucial step in biodiversity
analyses. Retrieved May 27, 2022, from
\url{https://methodsblog.com/2022/03/02/best-practices-for-taxonomic-harmonization-an-overlooked-yet-crucial-step-in-biodiversity-analyses/}

\leavevmode\vadjust pre{\hypertarget{ref-Grenie_Community_2020}{}}%
4. \textbf{Grenié, M.}, \& Gruson, H. (2020, July 15). Community
Captioning of rOpenSci Community Calls. Retrieved May 27, 2022, from
\url{https://ropensci.org/blog/2020/07/15/subtitles/}

\leavevmode\vadjust pre{\hypertarget{ref-Grenie_Access_2019}{}}%
5. \textbf{Grenié, M.}, \& Gruson, H. (2019, June 4). Access Publisher
Copyright \& Self-Archiving Policies via the 'SHERPA/RoMEO' API.
Retrieved May 27, 2022, from
\url{https://ropensci.org/blog/2019/06/04/rromeo/}

\newpage

\hypertarget{expuxe9rience-denseignement}{%
\section{Expérience d'enseignement}\label{expuxe9rience-denseignement}}

J'ai enseigné \textasciitilde130 heures en Écologie, mesures de la
biodiversité, et programmation R, de la L2 au doctorat.

Je suis formateur certifié de l'association
\href{https://carpentries.org/}{\emph{The Carpentries}} formant les
scientifiques à la programmation.

Au CEFE et à iDiv j'ai co-fondé des groupes d'utilisateur\(\cdotp\)rices
de R pour encourager une communauté de pratiques de R. \newline J'y ai
organisé et présenté plusieurs ateliers d'une heure sur des sujets liés
à R.

\begin{longtable}[]{@{}
  >{\raggedright\arraybackslash}p{(\columnwidth - 8\tabcolsep) * \real{0.4286}}
  >{\raggedright\arraybackslash}p{(\columnwidth - 8\tabcolsep) * \real{0.1429}}
  >{\raggedright\arraybackslash}p{(\columnwidth - 8\tabcolsep) * \real{0.1099}}
  >{\raggedright\arraybackslash}p{(\columnwidth - 8\tabcolsep) * \real{0.1758}}
  >{\raggedright\arraybackslash}p{(\columnwidth - 8\tabcolsep) * \real{0.1429}}@{}}
\toprule\noalign{}
\begin{minipage}[b]{\linewidth}\raggedright
Nom du Cours
\end{minipage} & \begin{minipage}[b]{\linewidth}\raggedright
Type de Cours
\end{minipage} & \begin{minipage}[b]{\linewidth}\raggedright
Année(s)
\end{minipage} & \begin{minipage}[b]{\linewidth}\raggedright
Niveau
\end{minipage} & \begin{minipage}[b]{\linewidth}\raggedright
Durée (éq.TD)
\end{minipage} \\
\midrule\noalign{}
\endhead
\bottomrule\noalign{}
\endlastfoot
Statistiques Descriptives avec R (UM) & TP & 2016, 2017 & L2 & 50h \\
Biogéographie Fonctionelle avec R (UM,
\href{https://github.com/Rekyt/functional_biogeo_practical}{support de
cours}) & TP & 2017, 2018 & M2 & 8h \\
Travaux Pratiques d'Écologie Fonctionnelle (UM) & TP & 2018 & L3 &
15h \\
Cours Invité sur les facettes de la biodiversité (UL) & CM & 2021, 2022
& M1 & 6h \\
Travaux Pratiques sur les facettes de la biodiversité avec R (UL,
\href{https://rekyt.github.io/biodiversity_facets_tutorial/}{support de
cours}) & TP & 2021, 2022 & M1 & 12h \\
Projet de groupe sur les facettes de la biodiversité (UL) & TD & 2021 &
M1 & 20h \\
Introduction à git et GitHub (UL,
\href{https://emilio-berti.github.io/idiv-git-introduction}{support de
cours}) & CM + TP & 2022 & Masters, Doctorant\(\cdotp\)es,
Postdoctorant\(\cdotp\)es & 10h \\
Formation diversité et rareté fonctionnelle (au CESAB,
\href{https://frbcesab.github.io/workshop-free/}{support de cours}) & CM
+ TP & 2022 & Masters, Doctorant\(\cdotp\)es, Postdoctorant\(\cdotp\)es,
Chercheur\(\cdotp\)euses & 4h \\
Cours Invité sur l'harmonisation taxonomique à
\href{https://www.nfdi4biodiversity.org/en/winterschool/}{l'école
d'hiver de NFDI4Biodiversity sur la gestion de données en écologie \&
évolution} & CM & 2022 & Masters, Doctorant\(\cdotp\)es,
Postdoctorant\(\cdotp\)es & 2h \\
\end{longtable}

\textbf{UM}: Université de Montpellier ; \textbf{UL}: Université de
Leipzig

\hypertarget{compuxe9tences}{%
\section{Compétences}\label{compuxe9tences}}

\hypertarget{langues}{%
\subsection{Langues}\label{langues}}

français {[}natif{]}; anglais {[}C2, courant et scientifique{]};
espagnol {[}B1-B2, quotidien{]}; allemand {[}A2-B1{]}; chinois
{[}A2-B1{]}; japonais {[}A2{]}

\hypertarget{languages-de-programmation-et-informatique}{%
\subsection{Languages de Programmation et
Informatique}\label{languages-de-programmation-et-informatique}}

\begin{itemize}
\tightlist
\item
  \faRProject~\textbf{R} ~\textbullet~ Meilleures pratiques de
  développement logiciel (tests unitaires avec\texttt{testthat}, site
  web \texttt{pkgdown}, intégration continue/déploiement continu,
  soumission au CRAN) ~\textbullet~ visualisation de données
  (\texttt{ggplot2} avancé et extensions) ~\textbullet~ manipulation de
  données spatiales {[}rasters \texttt{terra}, vecteurs \texttt{sf},
  base de données PostGIS, et interactions{]}
\item
  \faPython~\textbf{Python} ~\textbullet~ Programmation Orientée Objet
  ~\textbullet~ Tri de données avec \texttt{pandas}
\item
  \faDatabase~\textbf{SQL} interaction avec des bases de données
  ~\textbullet~ requêtes haute performance ~\textbullet~ interaction
  avec \texttt{R} ~\textbullet~ base de données spatiale
  PostgreSQL/PostGIS
\item
  \faServer~\textbf{Calcul de Haute Performance} soumission de
  \emph{job} sur SLURM/SGE ~\textbullet~ requêtes complexes
\item
  \faGit~contrôle de version ~\textbullet~ méthode git flow
  ~\textbullet~ développement collaboratif grâce à \faGithub~GitHub
\item
  Connaissances de base en \textbf{Julia}
\end{itemize}

\hypertarget{statistiques-et-moduxe9lisation}{%
\subsection{Statistiques et
modélisation}\label{statistiques-et-moduxe9lisation}}

\begin{itemize}
\tightlist
\item
  Modèles nuls et de permutations pour la biodiversité
\item
  Modèles mixtes généralisés
\item
  \emph{Machine Learning}/\emph{Artificial Intelligence}: \emph{Random
  Forest}, \emph{Support Vector Machine}
\item
  Modélisation de population structurées en stade, modèle de
  Lotka-Volterra \emph{via} Mathematica
\end{itemize}

\end{document}
